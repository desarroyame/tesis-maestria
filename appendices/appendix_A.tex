Juan Camilo Ahumada
Juan Carlos Arroyo

Entonces te acabo de contar, sé que en el 2013 tu hiciste este… o por lo menos se publicó digamos.

Ahumada: Sí, eso.

Se publicó el guion, y en el 2016 hubo un performance, y en el marco digamos de esta exposición, el distrito hizo un evento con respecto a Arte y paz, y ahí digamos en el marco de esto David Lozano, que es un maestro en artes plásticas fue de la Nacho, hizo otro performance y entonces como que este es mi corpus de investigación.

Chévere, pues digo chévere estar como en la periferia de esas cosas, pues yo no… aunque trabajo en torno a los referentes concretos visuales, musicales pues de corte estético, y también unos referentes teóricos obviamente pues al afrontar como un ejercicio creativo, en este caso los referentes eran cosas muy, muy, muy distantes al hospital, porque era una cosa que era de alguna manera una premisa cuando yo arranqué, yo quería hacer algo a unos testimonios, primero vi la cosa es así. Me disculpa ser tan escueto, pero este es el proceso de creación, yo vi unos videos en YouTube sobre el hospital San Juan de Dios, obviamente como inquieto por hacer algo en torno al hospital San Juan de Dios y como tratando de indagar en quienes habitaban el lugar, porque lo que más me interesaba bajo este tiempo en el que estuvo cerrado, y algunas personas decidieron irse a vivir allá. La obra gira en torno a eso, básicamente entre lo que pasaba entre estas personas ahí adentro, e hicieron un intento de documental que colgaron en YouTube que está en acabado, estoy hablando como en el 2005 tal vez, finales del 2008 no se, por ahí antes del 2009 en todo caso.
Y luego me surgió la posibilidad de hacer una residencia artística en Buenos Aires, con un maestro de escritura gramática que se llama Alejandro Tantanian, y cuando yo llegué a Buenos Aires llevaba todos los materiales, todos los testimonios, y una suerte como de caprichos estéticos.

¿Y esos testimonios como los recogiste? 

Tenemos los del video.

a los del video ok.

Luego yo tuve la oportunidad de conocer una señora que se llama Teresa Díaz, que lideraba en ese momento, yo no sé ni siquiera si la señora está viva porque no tuvimos ninguna relación, nos vimos una vez charlamos y ya, yo la grabe en video. Y pues ella estaba como muy interesada en hablar como del movimiento político que se había ido gestando ahí, el ejercicio de resistencia civil y esas cosas, pero yo la sentía como decirlo, como muy en el terreno de lo político.

Si.

Yo quería cuestionar la cosa desde otros niveles. También porque a nivel de lo político eso estaba completamente sesgado entre buenos y malos, y los buenos son los que están ahí y los malos son los que los quieren sacar, como en la estrategia del caracol y no es así, la cosa pensarla de esa manera sería aplanar de alguna manera el conflicto, yo sentía que ya estaba aplanando el problema definiéndolos entre buenos y malos, pero de ella pude coger como una suerte de voz, y una suerte de melancolía en el ejercicio de habitar este lugar y todo esto se articulaba a la perfección como con una cosa que yo quería tratar de materializar que era como unos universos de la desesperanza, y en ese momento este texto surge con otros dos, es decir como en el mismo momento y por la misma inquietud con otros dos que son absolutamente diferentes, uno que se llama llenasvenbrandy y el otro texto que se llama noche de alacranes, que son también como unas obras que tratan de darle vuelta a esto de la des-absoluta esperanza, como de la condición humana en ese momento del ocaso de la vida, como de esperar la muerte, como que la única posibilidad de esperar sería esperar morirse y entretenerse durante la vida, como sin ningún oficio concreto bueno… y esta obra se ganó el premio distrital de dramaturgia y entonces logro como sacar la cabeza por ese trio de obras, pero igual las otras están ahí, llenasvenbrandy la montamos y era como de alguna manera la preocupación que tenía por la desesperanza, partiendo de la desesperanza llegaron los testimonios del San Juan de Dios, y hablándole a unos amigos de este proyecto que yo tenía, alguien me dijo que conocía quien dirigía el movimiento pues de las personas que habitaban allí, que era doña Teresa Díaz yo me reuní con ella, y con estos testimonios me fui yo quería también hacer algo entorno como a lo trágico, pensando en términos teatrales, es decir quería hacer algo que estructuralmente se pudiera definir dentro de lo que es la tragedia en el teatro. Y para eso entonces me apoye de una ópera que es este kindertoten líder, que son las canciones de los niños muertos, the mahler, había otro referente ahí bien importante que ahora no lo recuerdo. Bueno pero había otro referente de corte musical, y tratando un poco como de… esos son los insumos concretos digamos las materias con las que trabaje, ahora mi aporte pues gira en torno al tratar de construir dos universos ahí entramos hablar de la obra, yo creo que la obra se sostiene como en dos universos, uno que son las cosas que suceden al interior de las personas, como el terreno del pensamiento, de la evocación, o del recuerdo bueno en fin, las cosas que les pasan por dentro a las personas, y otro plano que se contrasta como de una manera medio abrupta también ahí en la obra es lo cotidiano, y de alguna manera es como lo ordinario, el orden normal de las cosas de alguna manera lo vulgar, lo escueto, trivial si se quiere la lógica de la cotidianidad de una manera más… es que la palabra no es costumbrismo, pero la cosa si es como de una forma mucho más concreta, las cosas concretas los panes, las mesas, los dados, las cosas concretas de la vida de estas personas, y en el texto trato de armar. Originalmente este texto puede que esto de luces, originalmente el texto se llamó viacrucis y era una idea de hacer las estaciones, la idea digamos como estaba pensado era como unas estaciones en el espacio y que los espectadores pudieran transitar el espacio viendo estas situaciones instaladas en diferentes lugares, entonces hay una estructura narrativa que no es lineal y que se sostiene como en la construcción de estos pequeños universos que son casi que independientes. Y estructuralmente pues el recorrido quedo así como si arrancara, como una suerte de vocación, en la mitad está toda esta tensión con la cotidianidad, con el mundo concreto y terminamos como tratando de romper esta idea de la esperanza cristiana y del tiempo de Dios.
Ayer un muchacho en el bus, un ex drogadicto mencionaba el capítulo y el versículo de Ezequiel 3:3 que dice que los tiempos de Dios son perfectos y que todo tiene lugar y tiempo en el mundo de Dios y bueno estas cosas, que era de alguna manera como la justificación que ellos me daban o que yo sentía de ellos y es que el tiempo de Dios es perfecto y esto va a durar lo que Dios quiera de alguna manera, porque también hay que decirlo esto sucedió antes del gobierno Petro, cuando premiaron esta obra que fue en la mitad del gobierno Petro el problema se estaba disolviendo es decir si ese texto no lo publican en ese momento pierde completa vigencia, aunque la relación ahí no ha sido asumida tan directamente digo como por la gente que lee y eso.

No pero tiene mucho sentido es decir por ejemplo para mí sería muy difícil no reconocer digamos el momento en que estas obras fueron publicadas.

Si tiene mucho que ver con eso, claro

Hay algo de mis hipótesis pero tiene que ver con que el discurso digamos político social encuentra ahí un fenómeno en donde confluyen muchos de los discursos que de una manera a quienes estaban metidos en la problemática les servía el discurso político, se prestan mutuamente los intereses.

Si como un terreno fértil no.

Si.

Un terreno fértil como para que todos de ahí saquemos una buena cosecha seguramente fue así, porque también el momento en el que yo estaba en estas, estaba dándose esto con el gobierno Petro, como las discusiones, las conversaciones, la posibilidad de negociar el predio con la gobernación de Cundinamarca y esto, y un poco como se veía de lejos, porque yo pude ver además desde Buenos Aires como se veía de lejos la cosa, era pues si estamos abonando el terreno y ya luego todos, además todos incluso el gobernador de Cundinamarca que en ese momento era un hampón, todos sacaremos provecho.
Y ya el ejercicio literario pues fue un poco darme la pelea por lograr ser un poco coherente en la construcción de estos universos, y el texto está construido por unos hilos como narrativos que tratan de vincular los diferentes fenómenos. Pero si quiere hablemos de las escenas sí.

Por favor.

Ya hablando de eso de las escenas que cada una tiene un universo independiente, pues este primer cuadro que es solamente una acción, que este niño que está inflando unos guantes y está dejándolos pasar tenía un poco… se lo voy hablar desde la intención literaria de la teatral.
En lo literario yo quería como de alguna manera instaurar un tono para el texto, una manera de escritura, una textura para el texto de alguna manera como abriéndole la puerta al espectador hacia la manera en la que yo quería narrar la  historia, y creo que teatralmente espacialmente esto nos permitía como construir una metáfora del paso del tiempo, en ese momento una premisa era como una apertura hacia la eternidad, como hacia algo que no tuviera fin, bueno en fin, y esta figura del niño que va acompañando todas las historias o que las va atravesando pues de alguna manera era como presentar el anfitrión que va a ir con la gente haciendo el recorrido, pensándolo así digamos.

Si.

Este segundo cuadro que es el de Rosalía, es la presentación de este personaje, Rosalía es un personaje real que habitó con ellos en el hospital y que efectivamente la mandaron a habitar al edificio psiquiátrico, y que además es un personaje bellísimo, yo no tengo ningún vínculo personal con ninguno de ellos, pero si tuviera que tenerlo preferiría tenerlo con Rosalía, porque a pesar de que tiene una distorsión del pensamiento y que piensa de una manera paralela digamos a la convencional, me parece una mujer de una sensibilidad absolutamente particular y era la única digamos que estaba… voy a decirle el gesto y usted hace la interpretación, cultivaba y tenía animales dentro del lugar y se había dado una pela muy grande porque le respetaran el patiecito donde tenía sus matas y cultivaba, es decir no solamente matas sino cultivaba cosas y cuidaba unos animales pollos, perros bueno en fin, y ahí yo vi como un gesto más de empoderamiento como muchísimo más agresivo y eso le significó estar alejada de  los demás, de todos los demás que estaban habitando el Hospital, y además estaban en una pelea constante con todos muy fuerte. pero Rosalía también tenía un hijo, que es un hijo que luego se va a convertir aquí en el invalido un hijo con una discapacidad que yo nunca conocí, al que siempre me mencionaron y que siempre tuve ahí como en un ideal medio macabro de un muchacho quieto dentro de una pieza o dentro de un consultorio convertido en pieza, estático mirando una pared, contando ladrillos en una pared sin posibilidad de moverse ni de prender un televisor, ni de escuchar radio, ni de hacer nada en su vida más que ver y contar los ladrillos de una pared.
Bueno pero entonces aquí básicamente la presentación de este personaje, y tratar de ir instaurando el lugar, porque ella arranca como contando… describiendo quienes habitaban unos consultorios de consultas en un primer pasillo.

Este tercer cuadro es la presentación de Teresa Díaz, a quien le adjudicó el hijo de Rosalía, si es decir el hijo de la vida real de Rosalía yo se lo pongo a Teresa Díaz y la pongo como en un juego hay de amor y odio con el hijo en esta figura que tal vez está más adelante, hay una imagen en la que ella está afeitando al hijo acá, afeitar al hijo es cuidarlo, acariciarlo y también querer acabar con su vida que era un poco como esta cosa como tratar de construir esta imagen de la mamá que tiene que afeitar al niño inválido pero que en algún momento piensa jalar la cuchilla y dejar ahí, dejar ahí pues.
Esta es la presentación entonces de la enfermedad del muchacho, y aquí vuelve el niño y esta lógica como de la práctica científica y destacas en las que ellos fueron tan diestros de la… como de la  concreción de su oficio que tenía que ver con sangre, manipular carne, mover viseras, ajustar compresas llenas de sangre, botar restos de seres humanos estas cosas, aquí trataba de metaforizar por medio de un juego de un niño que es cirujano y que trata de abrir dentro de un muñeco y tratar partes y sacar partes dentro del muñeco mientras eso es amplificado en una proyección, de alguna manera quería que esto, la imagen que yo tenía era como de una cirugía que estaba sucediendo adentro y que se podía ver proyectada en el hospital como si quitaran partes digamos del espacio y adentro encontraron partes humanas, de alguna manera como una vuelta ahí a la imagen que era una imagen de referente con la que yo me había sido de un niño rompiendo un muñeco con un bisturí, trate de componer esa… además de esta imagen salió este niño que era vuelvo y digo el elemento que nos atravesaba como anfitrión, como se llama el Aqueronte que va con uno atravesando pues haciendo el recorrido como anfitrión.

Pero dices del hospital arquitectónicamente?

Si. Espacialmente, y además esto es una cosa que luego se volvió bien difícil, y es que yo escribí esto para ese espacio, pensando en ese espacio y nunca, nunca pude ni siquiera pensar en un proyecto creativo que tuviera que ver con eso y tuve que quedar satisfecho.

Pero recorriste ese lugar, digamos físicamente?

Sí, es decir está y hay una de las escenas que es esta donde están con el padre y que están haciendo como unas oraciones, está la del animero, es un recorrido por un espacio real y yo trato de darme la pela por describirlo, aunque pues es que yo soy muy mal autor entonces no lo logró describir a cabalidad, pero la idea si es que aquí quede el espacio todo esto del ritual del rezo y tal, aquí entre nos pues digamos es una manera de construir el espacio, y es una manera además densa y aburrida porque esto estructuralmente supone un momento que la obra pare como después de ir en quinta, mandar a primera de una y que toca haga como… y vuelva a instaurarse en otro ritmo muchísimo más lento.

Ya.

Que es un problema digamos para el ritmo de la obra, pero era necesario porque yo quería el recorrido espacial, es decir quería la gente en una procesión con esta lógica además del animero paisa que no se si conoce esta historia del animero, de un tipo que quería tal cual un poco como animero coro de actores haciendo de almas y espectadores transitando la representación.

Guiados por ese animero.

Si, si y construyendo el espacio a partir de eso, y había otro referente importante en esas tres obras además, yo creo que también un poco en mi manera de escribir y mi manera de articular las historias o de contarlas y es la cultura popular, obviamente distanciado del teatro popular y esas cosas que no tienen nada que ver con esto, pero si quería un poco tratar de también cuestionar esto desde la cultura popular, es decir esto como se pensaría también la nostalgia si el único referente posible fuera Alci Acosta.

Aja ok.

Si como ponerles palabras al dolor si el único referente es Vicente Fernández, por decirlo de alguna manera, o Roberto Carlos que aparece acá tratando de escribir una relación amorosa con un tipo que solamente escucha Roberto Carlos que es un personaje muy indo.
Y ahí empiezan aparecer este tipo de cosas, por ejemplo como esta escena es la escena número cinco de Edelmira, en la que hay como una suerte de rezo en la que ella está sola, en un espacio como con unas imágenes sagradas, y está tratando de construir como una suerte de rezo como de oración que se intervenía por este tipo de cosas que son los Visconti además, aquí juego como con las palabras de la oración y de la canción, el editor muy inteligentemente divide las cosas y  les pone negrilla a una cosa y a la otra y tal, como un poco para profundizar más en el guiño que fue un gesto que a mí me pareció muy bonito.

Empezamos hace un rato, en este momento Juan Camilo digamos me está contando a través de los cuadros, tanto la intención como un poco el gesto en el proceso de la creación.

Si, se fue digamos como materializando cada escena, yo estaba diciendo que cada escena funciona de alguna manera de forma independiente, como un solo universo dentro de una lógica estructural que se propone como un recorrido.
Cada una de estas escenas significa un espacio y  significa como una línea de acción que se cierra a sí misma, es autónoma pensándolo en términos narrativos, que es autónoma, una historia que se cuenta completa digamos, no necesariamente una historia con conflicto y solución y estas cosas, pero es una historia autónoma completa. Luego acercándonos ya a la mitad un poco de la pieza está este cuadro que es el del inválido que es el que yo creo que he logrado materializar de manera muy clara la intención inicial que yo tenía que era esta de poder ahondar lo que pasa por las cabezas de las personas que perdieron la esperanza por completo, estas personas que están un poco… yo decía ahora como en el ocaso de la vida bueno en fin, y este es un cuadro en el que se tratan de describir unas acciones, hay una contradicción pues planteada de entrada, y es que yo propongo un texto de cuatro páginas y digo que es un texto para ser bailado.

Ok.

Para ser bailado, para dar cuenta de la imposibilidad del  movimiento, y también un poco como insistiendo en esa incoherencia, pongo al muchacho sin palabras, el muchacho que no tiene acceso al lenguaje lo pongo a llenar de palabras, las sensaciones, estado en el que está, un momento muy específico en el que él está mirando en el patio, está mirando una pared contando ladrillos y viendo bichos que se cuelan entre los ladrillos, tratando de dar cuenta de la inmovilidad desde unas contradicciones, como tratar de contradecirlo para evocar esta inmovilidad, y obviamente también hay como unos gestos ahí, como unos guiños con estas cosas por ejemplo como de denominar lo inválido y no ponerle otro nombre con toda la fuerza y la agresividad además que tiene esa definición, invalidar deberás o anularlo lo que no vale, la silla, la silla que se pueda acomodar, y también pues esto responde a mi interpretación de esa relación de la mamá y del hijo, del hijo que ni siquiera decidió irse a vivir ahí, sino que fue porque tenía que estar con su mamá.
Esta primera escena que es del otro plano.


Si, a mi me llama… mientras la pausa para que no se me escape un poco lo que… y es que digamos te contaba un poquito en el correo cuando hablamos por el correo no, por el chat te contaba que digamos para mí esta pieza frente a todas las del corpus que estoy investigando es la que digamos me causa dificultad, pero ahora que te escucho empiezo a encontrar unas cosas, y es que finalmente yo me paro a hacer digamos un estudio sobre la imagen. 
Entonces cuando yo la leí digamos que evidentemente para mi tanto por cómo lo estás narrando y como información digamos como artista plástico digamos que yo me pinto el cuadro no, empiezo a leer y yo me pinto el cuadro. Ahora tal como lo cuentas encuentro que si efectivamente hay digamos una intención fuerte de construir una imagen de un momento específico.

Sí, y también de alguna manera yo pensaba el texto como un indicio sabe, que es una cosa bien distinta a los otros ejercicios de escritura, y es que esto pretende ser como un indicio, una pista sobre cómo construir espacialmente la obra, porque si creo que la cosa se escapa de la narrativa convencional teatral, sí es decir el teatro casi que se podría definir por la acción dramática, por lo que hacen los personajes, en este caso de alguna manera la obra si está planteada como un cuadro vivo digamos, en estos cuadros vivos de los católicos que es un cuadro con pequeñas acciones que no tienen mucho desarrollo, que es una cosa muy pequeña, pero que queda un planteamiento absoluto, absolutamente claro casi que de una época, de unos personajes, de una relación y de un espacio específico.

Son acciones que tienden a ser repetitivas, en esos cuadros vivos…

Si.

Casi ellos lo que construyen corporalmente es un ciclo, repiten un movimiento.

Eso es, y en esta lógica además del recorrido era un poco como yo pensaba el trabajo de los actores para asumir esto, y era de manera cíclica, es decir que una escena pueda terminar con el comienzo de esa misma escena, y esta escena que estamos… que vamos hablar creo que da cuenta de eso, perdón no es esta es la segunda de este mismo universo, que es esta con la que empieza con los tipos que están jugando dados, cuatro tipos uno que se encargó de oficios generales, el otro del transporte bueno en fin, que están jugando dados y la escena empieza con esto de anoche llamó el abogado que dice, seguro que esta semana sí, que eso está de un pelo, así empieza la escena le dan toda la vuelta, que la vuelta también es como en medio espiral, para terminar con esto, anoche llamó el abogado que dice, seguro que esta semana sí, que eso está de un pelo, más o menos, malparidos tal cual como empezó, tal cual, como dejando claro que es que la lógica si es circular, como no hay desarrollo, y eso va un poco en contra de la manera de escribir más teatral que necesitaría un desarrollo, y que sus acciones lleven a algo, que se articulen dentro de la noción de progresión dramática ya que esa progresión no existe, porque quería era que ahondáramos en ese universo de esa pieza de estos hombres que trato de escribirlos además como con detalle visualmente, cómo están vestidos, con que materiales, bueno en fin.
Y bueno pues estas dos escenas plantean dos cosas, como dos fenómenos que a mí me parecían necesario mencionar, y uno tiene que ver con el hambre y lo que pasa con la comida en estas situaciones, y la comida como la cosa más ordinaria, vulgar, cotidiana, concreta, humana pues comer, y la importancia que tiene comer y tener comida y cómo esto se chocaba esta cosa sobre la comida y sobre la ausencia de comida del padre y de la hija, de la plata y bueno de esta cosa, se contrasta con la discusión familiar que supuso tomar la decisión de irse, por un lado y por otro lado la anécdota que también ustedes conocen sobre la ambulancia que llevaron al INPEC.

Si.

Que era una ambulancia que ellos utilizaban para ir a abastos, que ese es como el subtexto de esto porque eso no está aquí solamente decimos que se llevaron la ambulancia del INPEC y que son unos hijueputas, y que se la trataron llevar otras veces y tal, porque deberás hubo ahí todo un operativo para sacarles la ambulancia, según ellos lo cuentan, ellos tenían que hacer vigilancia y una vez que los emborracharon y los cansaron y tal, le sacaron después la ambulancia con la complicidad de los celadores que siempre estuvieron riñendo con ellos bueno en fin, entonces esta ambulancia era importante y que se la había llevado era importante yo lo vine a comprender mucho después, era porque en ella podían ir hasta abastos y en abastos les regalaban comida, por eso me parecía importante dejarlo en este momento estructural que les mencionaba yo ahorita, como es el de lo concreto, el de lo operativo de alguna manera, lo que hay que hacer para que esta cosa funcione, y pues una grita también que para ellos supuso una grieta en el movimiento y una cosa muy fuerte esta cosa de que les quitaron, les quitaron de alguna manera una conquista que han logrado que era también tomarse este marica carro que todavía serbia.
Luego viene este recorrido espacial el del animero que tiene que ver con esta cosa paisa yo decía que también además es como un fenómeno paisa muy amarrado a la violencia en Colombia, y a los fenómenos violentos en Colombia, esta cosa de ponerle nombres a las almas y a esto que paso también el Valle fue, Trujillo Valle.

No, no sé.

Un pueblo que se tomó unas víctimas, iban encontrando víctimas por el río, hom,bres y los fueron sepultando y les ponían nombre y después les iban pidiendo favores, después de estos favores, de que estos favores fueran concebidos les iban dando apellido, el de su familia y los iban metiendo de alguna manera a su familia les decoraban la tumba, las flores unos N.N completos, y esta comunidad también hacia esta práctica de animero de un día al año ir y sacar a las almas a darles una vuelta por el pueblo y devolverlas otra vez para el cementerio, y tratando de pensar en eso pues aparece esta escena que también aquí en esta obra nos permite poner al espectador  a pisar las baldosas, a ver el piso, a ver el color de las paredes del hospital un poco como sin ningún tipo de  tratamiento estético que era el planteamiento, era andar el espacio y dar cuenta también de esta esperanza tan cristiana pero yo decía, o esta interpretación tan cristiana de la esperanza de que el tiempo de Dios es perfecto y que vamos a estar aquí el tiempo que Dios quiera pues eso es además lo que le da el nombre a la pieza, esta cosa de Ezequiel 3:3 de que en el mundo de Dios todos tienen su tiempo y todo tiene un tiempo y que bueno yo no sé cómo es que es que diga exactamente, pero básicamente dice que Dios es quien decide el tiempo en el que suceden las cosas de los hombres…

Y es perfecto.

Y pues así funciona la cosa, que hay que callar y agachar la cabeza y pues entender eso como la posibilidad más esperanzadora que tienen, pues es bastante desesperanzador digamos.

Bien y aquí cuando aparece este cuadro de las voces, pues es un intento por articular dos cosas, lo primero es la presencia del radio, del radio y del sonido del radio en el hospital mientras estas personas habitaron, cuando yo tuve la oportunidad de conocer el hospital fue una de las cosas que más me impactó como el contraste del sonido de los radios, yo fui en una tarde y en la tarde había tres programas de radio distintos en diferentes lugares, pero cada uno tenía su radio y su voz acompañándolo y nadie estaba pendiente de eso digamos era como un sonido de fondo en el hospital de alguna manera, yo quería materializarlo con unas voces que estaban dentro del hospital, unos tipos que estaban emitiendo el programa de radio, y el programa de radio es básicamente una descripción de diferentes fenómenos, este primero hombres rajados por la selva tal, masacre de las bananeras primer momento, masacre de las bananeras y esto tiene que ver también con el origen del hospital y la trayectoria del hospital quería dar cuenta por este programa de radio por todos los momentos de la historia nacional que el hospital de alguna manera testiguo, lo protagonizó algunos casos.

Si.

Entonces paso ahí por la masacre de las bananeras, más adelante el Bogotazo y Gaitán, la violencia y la lucha liberal conservadora y todo el fenómeno de la violencia posterior al Bogotazo, luego la toma del palacio de justicia y luego el narcotráfico, donde está la cosa sobre el narcotráfico, hay un momento en el que mencionan los aviones, que los aviones detengan su marcha pensando en ese atentado de los narcotraficantes al avión éste, y la explosión del avión y posterior la toma del palacio de justicia, a bueno y terminamos con esta cosa del narcotráfico de los precios, de cuánto vale todo, de que todo tiene un precio, de que pongámosle precio a todas las cosas.
Y bueno nos quedamos en eso de la violencia de los cincuenta, con esto que también son guiños a estos teóricos que yo mencionaba	 al comienzo, y casi que teóricos cliché de la violencia en Colombia por decirlo de alguna manera, muy chambona pero por decirlo de alguna manera como María Victoria Uribe por ejemplo, tú tienes el texto que se llama matar rematar y contramatar en el que describe los cortes, los cortes específicos y los mecanismos específicos de cada uno de los bandos durante el estado de violencia bipartidista, entonces por eso esta cosa de que quiero tener la cabeza entre las piernas, quiero tener un corte de franela, quiero que saquen mi lengua y que la cuelguen como una corbata, es un poco como unos guiños a esta manera a esta mujer de describir la violencia a partir de los mecanismos, y si al final cerramos pues con paramilitarismo y masacres paramilitares con esto que era también una preocupación muy grande en ese momento que trata de materializar llenasvenbrandi que es la masacre del Salado, como tratar de entender cómo funcionó de alguna manera la masacre del Salado, era otra cosa que me preocupaba en ese momento como una escritura paralela que estaba haciendo a esto, por eso también aparecen estas figuras que pues la imagen trata de definir un poco la masacre del Salado, los vi llegar en enormes bestias, devorarnos, quitarnos los brazos y las piernas, los vi arrastrar mi cuerpo alado por caballos, los vi correr, hacer música con tamboras y gaitas mientras me desmembraban quiero respirar un minuto de silencio bueno, esto pues son imágenes que yo creo que todos quienes conocemos un poco de lo que paso pues tenemos en la cabeza y es esta cosa de la mujer jalada por caballos y destrozándose su cuerpo por todo el pueblo, los hombres jugando futbol.

Y la música de…

Y la música… y estos tipos que van y se meten a la iglesia que sacan las gaitas y los tambores mamados de los gritos, borrachos empelucados empiezan hacer música mientras degollan o juegan futbol con cada vez...buen en fin.
Y esto ya para ir cerrando que es el cuadro de Rodríguez que es volver otra vez a ese universo de lo poético y vamos como más…

Mental.

De lo mental, de las evocaciones, y es presentar de alguna manera este personaje en esa misma tensión como por buscar que la cultura popular logre definir lo que este tipo siente, entonces este tipo se llama Rade Roberto Carlos que es lo que más ha querido en su vida y que el además se cree cantante que canta como Roberto Carlos Y bueno en fin, y muestra un cuadro que tiene un tono bien costumbrista, bien cotidiano, muy tranquilo, pero que muestra también un personaje que navega pues, como que circula, que bucea como en cierta ignorancia y  que trata de interpretar el mundo desde su ignorancia y desde su… voy a… si es que soy muy grosero con el lenguaje pero… desde su miserable manera de ver las cosas él trata de mencionar como los momentos en los que se ha sentido más pleno, más lleno de vida, y pues eso tiene que ver con el amor y con la construcción afectiva que pudo construir, hacer con una mujer, y como esa utopía alcanzada se vio rota en el momento en el que él tuvo que decidir me voy para el hospital, y que la tuvo que dejar y ahí más nunca volvieron hablar y la cosa se rompió como tratando también de dejar una historia de amor inconclusa, tratar de dar cuenta un poco del vacío en el mar de la desesperanza, de la ausencia, porque él mismo dice si yo me le aparezco yo no sé ella con que me saldría a estas alturas del partido, y pensándolo también en términos del señor pues digo el señor de 60 años, de 65 años que todavía piensa en su mujer como su amor, pero que ya no sería capaz de presentársele porque sabe que probablemente esté con otro tipo y la cosa sería muy complicada y bueno en fin.

Aquí aparece el cuadro de amar que se amarra con ese, de la afeitada del hijo, y del conflicto Teresa y su hijo, Teresa contempla a su hijo con algo de desesperanza y su hijo la ve desvanecerse en medio de una nube de humo tratando como de concluir un poco como este gesto con el que ella inició, y aquí aparece esto the mahler que es una ópera muy trágica, que se llama canciones de los niños muertos y que básicamente es un poema que escribió un papá sus niños como a finales de 1800 se murieron de una enfermedad pulmonar, digamos tuberculosis, y el tipo compuso una serie de poemas todos terriblemente trágicos y dolorosos en término de lo que significó para el perder a sus hijos y perder con ellos la esperanza de ser alguien, porque él había dejado toda su vida por sus hijos y a sus hijos se los llevó un virus, este cuadro trata un poco de dar cuenta esa misma sensación pero puesta en Teresa y viendo la imposibilidad de dar un otro estado a su hijo, de producir otra cosa, otra vida para su hijo y de ver acabada pues su esperanza que se debería materializar por medio de él verla rota.

Bueno, Edelmira y volvemos a Edelmira que es este personaje del lío mental que mencione antes, tratando un poco Edelmira de continuar en esta misma lógica que es la lógica del inicio, de cerrar de manera circular de cerrar su círculo, y ella cierra su círculo siendo en caso de que llegue a solucionarse la cosa ella lo que más esperaría de la vida sería irse para ese apartamento un poco como esto, desde aquí alcanzo a ver ese apartamento y me he emocionado tanto con él, que en caso de que yo me llegué a ir de aquí yo quisiera irme a vivir allá, que es un poco como dar cuenta de la imposibilidad y del poco mundo que había ahí que es ella mirando desde las terrazas del hospital hacia el Policarpa o hacia el samber es decir o una olla o un barrio proletario de donde venden telas, obviamente el barrio de telas es un lujaso pues un lujaso comparado en el momento en el que estoy y con la otra posibilidad que tengo, esta era el mundo de alguna manera que ella podía asimilar y yo trato de cerrar... 

El mundo al alcance de sus ojos.

Claro, con ella diciendo eso que el mundo llega hasta donde yo miro como en el feudalismo medieval, el mundo va hasta ahí hasta donde me alcance el ojo, y este cuadro también pues obviamente tiene la intención de cerrar la intención pesimista de cerrar en ese mismo plano pues desesperanzador, desalentador de encontrar a un en la esperanza de estas personas pues como una misión muy miserable por decirlo así, una ilusión muy mínima, muy escasa.

Ella trata de comparar eso además con la casona con la que vivió con su papá que fue magistrado, que tuvo otro estado, con el que tuvo otro tipo de vida en el que fue muy feliz, pero entendiendo la felicidad también como Rodríguez, entendiendo la felicidad como una cosa distante, ajena, perdida en el pasado de alguna manera y pues cerramos con este pequeño monólogo que es femenino el texto no dice más, es como un gesto del autor para cerrar es un gesto mío que intenta cerrar con voces femeninas o con una voz femenina describiendo este texto yo lo voy a leer.

Anoche soñé con olas, he empezado a mortificarme por lo que fue, por lo que pudo haber sido y no fue, eso es como la premisa para todo este texto, me lamento hasta en sueños no conozco el mar, supongo que alguien debe sufrir por una razón más justa, más profunda pero yo no, es la idea como de poner al personaje a mirar su esperanza, verla rota, reconocer y entender que su esperanza era mínima y la fractura que supone ver esa esperanza diluirse pues tampoco puede ser tan grande pero es su sufrimiento mayor, y era su ilusión mayor de alguna manera, y quería también ponerlo como en unos términos muy sensibles pero también muy aterrizado a la tierra, es decir cómo ponerla realmente a lamentarse por una cosa que es muy distante a nosotros los Bogotanos, que ha sido distante a demás tradicionalmente esta cosa de la tierra caliente y el mar, todo lo que significa para nosotros el mar, para nosotros que estamos tan lejos y lo evocamos pues, y que hablamos de él  todo el tiempo.

Siempre es importante, la primera ida al mar del rolo es importante.

Claro que sí, y esta sensación de las piedritas que se mueven por debajo de los pies ya que tocas eso, pues tiene que ver con mi primera sensación al mar, que yo recuerdo pues iba en una lancha para un lugar cerca de Acandí y me baje de la lancha ya en el momento en que se bajaron a dejar unos bultos y lo primero que sentí fue que el agua se llevaba las piedras debajo de mí y que yo me iba a hundir, esa sensación era como mi zumo para hacer esto, hoy mi dolor es solo ese, y bueno aquí entre nos tengo que confesarles que como recurso mío, propio para este texto pues estaba pensando en mi mamá, como pensaría mi mamá el mar porque siempre me impresionó mucho cuando yo crecí también cuando conocí el mar por cosas de trabajo y tal en fin, me empezó a preocupar de una manera muy rara que mi mamá no conociera el mar, como que hubiera alguien además cercano a mí que yo quisiera tanto tal, que no hubiera nunca tenido esa sensación, sin conocerla entonces la puse a ella a lamentarse por esa sensación que desconocía y a describirla con detalle, como esta de las piedritas. 

Me duele nunca haber escuchado el sonido de las olas golpear contra las rocas, o deslizarse llevándose la arena y dejando piedritas pequeñas desnudas en el suelo, que es una imagen bien precisa y la pongo igual a construirla desde la pura ignorancia de esta imagen, lamento no haber sido arrastrada por una fuerte marea hasta no sé dónde, nunca sentí las olas venir hasta mí y llevarse la arena bajo mis pies, nunca, me duelen las cosas que nunca me pasaron más que las que sí pasaron y quisiera olvidar, me duele aquí en el estómago el dolor va bajando hasta inmovilizarlo las piernas, nunca más estas piernas se moverán porque no ha sido arrastrada bajo ellas la arena del mar.
Y pues obviamente tiene el cierre, tiene que ver con la desilusión que es una manera tal vez mía pero también de muchísimos autores de negar la esperanza  también en términos estructurales, hay una cosa estructural en la obra que dice pues en la escritura dramática que dice pues que al terminar algo debe concluir o  algo se debe de haber transformado, este gesto tiene que ver como con negar esa posibilidad de transformación o de llegar a otro estado, sino de circular en el mismo momento en que empezamos, que es la inmovilidad, casi esto es imposible de hacer así porque sería aburridísimo, pero casi que si me preguntan yo me imagino a los personajes a todos sentados y moviendo solamente la boca, es una manera como de ver la obra, como cuando la fui escribiendo era una manera de componer también las palabras era pensándolo así, un personaje quieto o con una pequeña acción tratando de decir, esta fue una frase bonita que nunca quedó en la obra es de esta obra, tratando de decir mi nombre es Teresa Díaz, tengo 58 años y un hijo eso es todo lo que tengo y dejarla en silencio dos horas más, pero encerrada con el público, y dejarlos ahí como con esa frase en la cabeza tratando de interpretar el mundo desde esa lectura que ella también se hace de ella, y es que lo único que he logrado en la vida es cumplir años y tener un hijo no hay más.
Bueno eso es lo que yo tendría por decir como exposición, ustedes quieren preguntar algo, o hablar de otra cosa, todavía nos queda hacer de eso.

Bueno hay una cosa que tu pones de manifiesto en la obra y como no lo estas contando ahora y es lo que digamos yo estoy intentando cruzar en el ejercicio en términos metodológicos de investigación...

Si.

Es que usar algo que resulta bastante complejo y es como yo cruzo la imagen en este caso imagen artística con el discurso oficial, institucional, político de los medios y es que mientras unos les jugaban digamos a la transformación, la transformación como oportunidades de salida, la transformación del hospital como una oportunidad de salida quienes resistieron y sobre todo en esa forma particular de resistencia de los que se quedaron viviendo porque fueron distintas maneras de resistir, esa era una.

Claro, la más radical digamos.

Si era una de varias maneras de resistir, pero pone de manifiesto digamos para mí la única salida es la continuidad de lo que yo conozco, de lo que ya tengo, de lo que siento que merezco y que empiezo a cargar en la medida de mi lucha una supuesta necesidad social de que el hospital San Juan de Dios tal como es, tal como era es lo que debemos seguir teniendo, y ahora que tu contabas del rol de los buenos y los malos, digamos que yo entreviste a uno de los malos buenos, es decir a uno de los trabajadores de la entidad, distrital que estaba digamos intentando que la gente que estaba viviendo se fuera a otro lado…

Yo sé, y eso fue un mierdero.

En muy buenos términos y digamos ayudarlos, pero para ellos no representa directamente una ayuda.

Claro no, eso era casi un desalojo.

Exacto, yo les quiero transformar la vida, y ellos no me la transformes.

Si, dame lo que te pido o déjame, si y yo creo que eso se cruza con un poco con eso que yo estaba tratando de decir de Edelmira, y es decir como obviamente que quiero estar mejor pero puede que para mí estar mejor es estar aquí adentro, es decir estar a un paso también porque  este es el mundo que yo conozco, y obviamente también habla de unos universos súper reducidos y unos discursos maniqueos, manipulados que se repiten y se repiten como asumiendo las verdades absolutas, como pasa en últimas con los discursos políticos que es como una suerte de verdad que… como una suerte de mentira que a fuerza de decirse tantas veces va convirtiéndose en verdad.

Y en esto del discurso político queda expuesto totalmente porque la reapertura se anuncia constantemente en distintas versiones no, en el documento oficial, en la visita protocolaria.

En la inauguración.

La inauguración no, vienen lo abren viene el presidente y abre el hospital, inclusive con el paso al siguiente gobierno distrital de Petro a Peñaloza, la primera postura del discurso de Peñaloza frente al San Juan es ningún lo estamos abriendo, el hospital está abierto y mírenlo acá y intenta construir un discurso de la apertura, pero es una apertura relativa, es una apertura que no es esa apertura.

Que nos es lo que se imaginarían, además es que obviamente quisieran tener, pues quisieran que el Hospital San Juan de Dios fuera el más importante del país y bueno en fin, que  recibiera a los estudiantes de la Nacional, pues como estuvo a comienzos del siglo, pero también eso se choca con unas cosas concretas que hacen que eso sea imposible, el solo planteamiento de la ley 100 que es como el ordenamiento legal del sistema de salud en Colombia, imposibilita ese sistema de organización que ellos proponen, no habría manera, pero sí creo que esos discursos se legitiman de una manera poco violenta pero muy eficaz, muy eficaz es decir se van instaurando como a fuerza de decirte tantas veces pues hay dentro había como una sensación que había un discurso oficial que había que asumirlo com... pues de manera dogmática, es decir es así, nos tiene que dar tanto no tienen que indemnizar, nos tiene que pagar lo que nos deben y además nosotros estamos defendiendo una cosa que es pública, es decir además esto lo tienen que abrir en estas...

