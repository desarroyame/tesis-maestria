
## Contextualización espacial

Enmarcado en el centro geográfico del casco urbano de Bogotá se encuentra el Hospital San Juan de Dios, ubicado hacia los cerros sur orientales, en la localidad 15 Antonio Nariño, entre las calles 1 y 1a sur y las carreras 10 y 14 (Avenida Caracas), se encuentra el imponente lugar donde se salvaron muchas vidas, y en donde hoy día se hace evidente uno hito de resistencia de nuestro país. Sobre una superficie de 13 hectáreas en las cuales hay 24 edificios incluyendo el Hospital Materno Infantil, en donde existían 14 salas de cirugías y 500 camas que atendieron a muchos nombres y mujeres hasta el año 2001. Dicho sector, está caracterizado por ser un espacio en donde se encuentran símbolos urbanísticos de la ciudad y de gran importancia para la historia local y nacional. Cabe destacar, la cercanía con el parque Tercer Milenio, la iglesia del Voto Nacional, barrios tradicionales como el Policarpa, San Bernardo y Eduardo Santos, del mismo modo, se hace evidente una gran articulación a nivel hospitalario al estar vinculado a otros espacios de salud como el Hospital Santa Clara, La Samaritana, El Hospital de la Misericordia, entre otros. En el mismo sector hay zonas de naturaleza residencial y comercial; prima la industrial textil, mecánica y de fabricación de muebles y enceres.

Sin embargo, su ubicación no fue siempre allí, durante la colonia en el siglo XVI el Rey Español ordena que se construyan hospitales para atender tanto a naturales como a españoles, es así como surge el primer hospital de Santafé, cuyo primer nombre fue San Pedro, de Jesús, María y José y se ubica en dos espacios: En primer lugar, _“(…) Fray Juan de los Barrios y Toledo otorgó escritura pública …. Donando unas casas de su propiedad situadas en la calle de San Felipe (hoy carrera sexta).”_(Soriano, 1964, p. 8) Exactamente detrás de la catedral primada, en donde contaba con apenas doce camas, sin embargo, cuando este espacio ya no dio abasto se tomó la decisión de ampliarlo y cambiarlo de ubicación hacia 1723 que _“con el producto de la venta de varias casas que pertenecían al Hospital, desde su primera fundación, y con buena parte de las limosnas recibidas, se inició la construcción de la nueva sede bajo la dirección de Fray Pedro Pablo de Villamar”_ Era una edificación ubicada en la calle San Miguel que hoy día son las carreras novena y décima entre calles once y doce de Bogotá. Eventualmente, fue trasladado a los predios del “Molino de la Hortua” el cual era una finca cuyo propietario era José Domingo Ospina.



Ilustración 5. Ubicación del HSJD

## Contextualización histórica

Los orígenes del complejo hospitalario San Juan de Dios se remontan al siglo XVII, por las épocas de La Nueva Granada, cuando las órdenes religiosas que llegaron a América eran las encargadas de coordinar los distintos aspectos de la vida cotidiana tales como educación, salud y por su puesto lo espiritual. “_Aparece la adecuación de antiguos edificios para funciones hospitalarias, definiendo un comportamiento en el modo como se sitúa la práctica médica dentro del contexto de las sociedades urbanas y su devenir histórico”_ (Romero Isaza et al., 1994, p. 15).

Durante el año 1635, el arzobispo Fray Cristóbal Torres encargó de la tarea de administrar los espacios hospitalarios a la comunidad de San Juan de Dios, en cabeza de Fray Gaspar. Lastimosamente, dichos espacios no fueron suficientes para atender la demanda de los enfermos tanto españoles como criollos, a lo que la sociedad bogotana reclamo, y las autoridades eclesiásticas respondieron a dicho reclamo mediante la donación de un terreno ubicado entre las calles 11 y 12 y carreras novena y décima, y se llamó “Jesús, María y José” Montero (Fajardo, 1994, p. 166).

“(…) la naturaleza de este hospital respondía a la de un hospital episcopal, lo que quiere decir que era una fundación protegida y sometida al patronato del obispo, para atender a los religiosos de la Provincia y a los pobres, en las mencionadas estructuras arquitectónicas adecuadas para tal fin, y que perduro hasta 1739, sobrellevando la caducidad y estrechez de los edificios, la reiterada negación de la administración colonial para sufragar su manutención y el permanente aumento de enfermos” (Romero Isaza et al., 1994, p. 18)

No fue sino hasta 1735 que tomo el nombre de Hospital San Juan de Dios, y empezó a funcionar hacia 1739, manteniéndose económicamente mediante la figura de “obra pia”, que no era otra la de cambiar bienes materiales por una entrada al cielo, posteriormente hacia 1790 el hospital sirvió como espacio para la atención de las víctimas de la insurrección de los comuneros; y de las pugnas independentistas, del mismo modo, la población urbana aumento densificando el espacio, aumentando así el número de usuarios generando que los religiosos abandonaran la administración del Hospital.

Durante 1868 nace la Universidad Nacional de Colombia y en 1869 la junta administrativa del Hospital de Caridad y de la Junta General de Beneficencia de Cundinamarca asumieron la responsabilidad de administrar el Hospital San Juan de Dios, _“en 1900 el hospital contaba con tres salas de operaciones, dos pabellones de clínica externa de hombres y mujeres y un servicio de ginecología, el cual se independizó totalmente de cirugía en 1903”_ (Fajardo, 1994, p. 167), en 1901 bajo la necesidad de un hospital universitario nace en sí el San Juan de Dios, durante 1912 mediante la ordenanza No. 37 del departamento de Cundinamarca se establece que el Hospital pertenece a la beneficencia de Cundinamarca haciendo que se impulsaran obras para la mejora de la atención en el mismo, en 1913 se inauguran espacios como el “manicomio moderno” (Romero Isaza et al., 1994, p. 46), en 1936 se hace necesario construir el camellón que comunique los edificios; además de esto se construyen el “_pabellón sur, destinado a la clínica de enfermedades tropicales, y los dos pabellones occidentales donde quedaron instaladas las clínicas médica y semiológica”_ (Romero Isaza et al., 1994, p. 109)_._ La consolidación e importancia del Hospital económica, social como científicamente se da gracias a situaciones como el apoyo de la Universidad Nacional, antes mencionado, el descubrimiento de la hidrocefalia de presión normal por Salomón Hakim en 1957, el desarrollo de la primera vacuna contra la malaria en manos de Manuel Elkin Patarroyo en 1987, la implementación exitosa del programa mamá canguro y por supuesto la atención que brindaba a distintos ciudadanos sin importar su condición económica y en momentos coyunturales de la historia del país y por supuesto de la ciudad.

Sin embargo, el fantasma de la decadencia ya empezaba a rondar el San Juan de Dios, durante el año 1975 se lleva a cabo la primera huelga por parte de los empleados, que empezaba a vislumbrar lo que sería el futuro, y es durante el año de 1978 cuando el Gobierno Nacional se aparta de la administración del hospital dejándola en manos de la gobernación de Cundinamarca, cobijándolo bajo la figura de “Fundación Privada” , haciendo que con esto, que en la nuevas reglas de nuestro país no se contemplara ni se tuviese en cuenta la inversión de dineros hacia el Hospital, o en otras palabras, como lo establece el Artículo 355 de la Constitución Política Colombiana, ya que _“Ninguna de las ramas u órganos del Poder Público podrá decretar auxilios o donaciones en favor de personas naturales o jurídicas de derecho privado”._ Por otro lado, para empeorar la situación del Hospital, fue la creación de la Ley 100 de 1993, en donde se establece la mercantilización de la salud; el Estado subsidiará la demanda y no la oferta, dicha situación sello el debacle del Hospital, cediendo responsabilidades a la Beneficencia de Cundinamarca, la cual no pudo asumir dichas responsabilidades ni competir con Clínicas privadas; es así, como desde 1994 el Hospital empezó a presentar problemas económicos acarreando con esto su liquidación en 1999 bajo el gobierno de Andrés Pastrana, quien lo justificó dicha decisión en la idea que el San Juan de Dios era una entidad privada, y por esto no le era posible dar recursos para su funcionamiento, declarándolo inviable. Es justamente allí cuando algunos empleados deciden abandonar las instalaciones, ese momento de la historia del Hospital se conoce entre los empleados que se quedaron como el día del “éxodo”.

Posteriormente, las labores del hospital continuaron a media marcha, durante el año 2000 las manifestaciones de desacuerdo por la decisión tomada por Andrés Pastrana se intensificaron, mediante protestas y acciones dicientes, sin embargo, estas situaciones no detuvieron el desenlace fatal a la situación del San Juan, hacia el 2001 las empresa encargada de la distribución de energía eléctrica de Bogotá, Codensa, le corto el servicio al hospital por una deuda de 2.000 millones de pesos, haciendo así que sus labores se detuvieran por completo y llevando a que el último paciente abandonará el Hospital.

No obstante, la situación no acaba allí; algunos empleados decidieron seguir en la búsqueda de una salida digna a la situación, y gracias a estas se logró que hacia el año 2002 las instalaciones del Hospital se declaran con patrimonio mediante la Ley 735 del 2002 la cual contiene cinco artículos, que además de declarar monumento nacional el San Juan de Dios también establece en su Artículo 3 que:

“El hospital San Juan de Dios y el Instituto Materno Infantil continuarán funcionando como un centro especial para la educación universitaria que imparta, en las ciencias de la salud, las universidades oficiales y privadas, esto es, como hospitales universitarios.

Para los efectos del inciso anterior, se considera hospital universitario aquella institución prestadora de servicios de salud que mediante un convenio docente asistencial, utiliza sus instalaciones para las prácticas de los estudiantes de las universidades oficiales y privadas en el área de la salud; adelanta trabajos de investigación en este campo; desarrolla programas de fomento de la salud y medicina preventiva; y presta, con preferencia, servicios médico-asistenciales a las personas carentes de recursos económicos en los distintos niveles de atención y estratificación”.

A pesar de todo esto, no fue suficiente para que el gobierno hiciera el esfuerzo para reabrir el hospital, pero si para que no fuese demolido, pero, por otra parte; en el año 2015 la administración distrital de Gustavo Petro demuestra su intención de volver a poner al servicio de la comunidad el San Juan de Dios y de hecho se lleva a cabo un acto simbólico de reapertura el 11 de febrero de ese año con el acompañamiento del presidente Juan Manuel Santos. Más adelante veremos que esta supuesta apertura es una de varias que no conducen a cumplir las expectativas de los trabajadores que durante años han esperado esa anhelada reapertura que al menos al término de esta investigación no ha llegado.

## Arte y visualidad en tiempos de crisis

Ente la prolífica producción artística y audiovisual que se dio en torno a este fenómeno sociocultural, y fundamentado en la antropología de la imagen y los estudios visuales esta investigación aborda la relación entre imagen artística y formas de mantener la memoria e imaginarios sobre el San Juan de Dios.

Se revisan los signos de las luchas por los derechos laborales, reconocimiento patrimonial, salud pública y el cuidado de la vida, a través de registros gráficos y audiovisuales de procesos de creación artística, instalaciones in-situ, performativas y de dramaturgia, que como veremos en el desarrollo de este informe, contienen signos icónicos e indicios de la imaginación social, las imágenes proyectan de forma no-textual algunas explicaciones, reflexiones y emociones sobre el caso del HSJD.

Un hospital en abandono y que se deteriora es una imagen poderosa, llama la atención, es como una erupción en la piel, es algo que a simple vista parece no estar bien.

Los medios comunicación se han apegado al trasegar histórico, haciendo reportería de las acciones de resistencia y propaganda de gestión institucional, sin conceptualizar ni profundizar en las múltiples discusiones, estos medios tienen efecto en la opinión pública sobre lo acontecido en el HSJD, _una opinión pública que está en un estado intermedio entre la ignorancia y el conocimiento_ (Neumann, 1995).

Durante el reciente periodo de crisis y cierre del HSJD varios comunicadores, artistas y realizadores audiovisuales se vieron motivados ante esta manifiesta y poderosa imagen de coyuntura, conflictos, intereses y robusta trayectoria histórica, patrimonial y social.

Cuando un lugar consagrado al cuidado integral del ser humano entra en crisis, y las personas que trabajaron allí salvando vidas se ignoran, no se puede simplificar a un caso administrativo o institucional, es un fenómeno de interés social, todo aquello relacionado con la salud es de interés público. Ese interés se hace evidente en las imágenes, muchísimos registros audiovisuales de reportería e investigación periodística han sido creados y circulados durante las últimas dos décadas. Alugas imágenes ilustran los hechos, lugares y personas, pero hay otras imágenes no buscan la representación, ilustración o comunicación del fenómeno, como se explicará más adelante, esas “otras” imágenes son las imágenes artísticas, y en ellas se expresan otros atributos, mensajes no-textuales sobre la complejidad de problemáticas, entramado de eventos y consecuencias del abandono que sufrió el HSJD y su comunidad de trabajadoras.

Comprender algo en nuestra contemporaneidad tan rica en imágenes y medios esta intrínsecamente relacionado con la acción de “ver”. Enterarse de un acontecimiento o situación de actualidad no es exclusivamente un acto de “lectura”, de hecho, podría decirse que ya pocos son los que acceden a la información exclusivamente a través del medio textual. Es más probable que se acceda a la información a través de alguna red social deslizando imágenes en la pantalla de un teléfono o en la pantalla de una estación de trabajo o entretenimiento.

El arte como referente simbólico colectivo señala o expresa la experiencia social, así, las imágenes resultado de registro de la producción artística y las experiencias estéticas en torno al caso del HSJD, aunque no se proponga como ilustración de los acontecimientos, ni correspondan a interpretaciones de lo que “realmente” aconteció, es decir la verdad sobre el caso del San Juan, sí condensan imaginarios del momento y el fenómeno sociocultural.

Según Mitchell una importante observación del semiólogo Roland Barthes es la resistencia de la imagen al significado, la imagen tiene atributos que construyen sentido en sus silencios, su reticencia, salvajismo y absurda obstinación (Mitchell William John Thomas, 2005).

La imagen artística o registro de acontecimiento plástico, estético, poético, es un vehículo o medio que tiene unas intenciones y permite cierta experiencia de interacción con lo que estamos viendo y por ende impacta en la representación interior de las problemáticas sociales complejas a las que hace referencia o es inspirada.

Veamos algunos ejemplos de relación entre imagen artística y crisis. Durante el decaimiento de Detroit las intervenciones de Tyree Guyton transforman y crean imagen en su entorno cercano, aquello decadente lo vemos como símbolo de retorno de la esperanza; en esta misma línea, otro ejemplo es el análisis de Julia Rothenberg a la obra de a varios artistas neoyorquinos en torno al 9/11, estas obras fueron una respuesta crítica a los acontecimientos históricos, y sus mensajes son tanto lo que tiene como la ausencia de aquello que las imágenes no pueden tener, señales desde sus silencios, ¿qué nos hacen estas imágenes?, ¿qué quieren las imágenes?

“Preguntar ¿Qué hacen las imágenes? No es solo atribuirles vida, poder y deseo, sino también plantear la cuestión de qué es lo que les falta, qué no poseen que no se les puede atribuir” (Mitchell William John Thomas, 2005).

Un ejemplo de esto es el caso de investigación a través de las imágenes es el trabajo de la socióloga Julia Rothenberg quién analizó las respuestas de varios artistas locales en Nueva York a los ataques del 11/9 representados a través de sus obras de arte (Rothenberg, 2002). Luego contrastó las representaciones de los artistas del evento con la representación del 11/9 a través de los medios de comunicación. Utilizando las estructuras teóricas de Adorno y Benjamín, y su _deseo de cambio social radical_, Rothenberg situó la obra de arte en ese concepto para explorar el marco de la experiencia del 9/11 (Leavy, 2018, p. 515).

En Colombia hay momento claves de crisis y resistencias sociales que también han llegado a nosotros mediante imágenes, registro de obras de creación artística. Algunos de estos acontecimientos son: masacre de las bananeras, el Bogotazo, la violencia bipartidista, la toma del palacio de justicia, el narcotráfico, las guerrillas y el paramilitarismo, por mencionar los más evidentes.

La historia del arte nos ofrece la oportunidad de reflexión a través de la imagen, y a través de ella transitar por los pasajes epistemológicos compartidos, los imaginarios sociales.



Ilustración 6. “La República”



Ilustración 7. “Musa paradisiaca”

La masacre de las bananeras fue un acontecimiento crítico en la historia nacional, en ese caso confluyen entre otros asuntos, macroeconomía, relaciones internacionales, política, sindicalismo y violencia. La ilustración #3 es la obra del artista Pedro Nel, que plasma en su pintura realista una mezcla de motivos naturales y las situaciones sociales del momento. Hace uso de textos para hacer referencias directas a los movimientos de protesta, se lee en el mural representaciones de pancartas con las frases como “La conquista del subsuelo, la obra para este siglo”, “2500 capas de subsuelo minero ya no pertenecen a la república” y “Defendamos la república”(Gómez, 2013). Es notable que no se destaquen los racimos de plátano, el evento tiene referencia directa con el cultivo, aún así la mirada del artista retrata un cierto imaginario sobre los actores sociales.

Señalando el mismo evento de crisis y encontraste con esa imagen realista vemos en la ilustración #4 una presentación de racimos de plátano para señalar un factor esencial del mismo fenómeno, haciendo énfasis en la guerra sucia de las multinacionales.

“Mi primer encuentro con la “Musa paradisíaca” fue a través de un pintoresco grabado del siglo XIX: una sugestiva mulata aparecía reclinada bajo una planta del banano”. Se trata de Musa paradisíaca, un grabado publicado en el libro “Viaje a Nueva Granada” del francés Charles Saffray, quien relata sus observaciones sobre la fauna, flora y costumbres locales en 1861. La planta del banano fue clasificada científicamente como Musa paradisíaca. Esta imagen y su título inspiraron a José Alejandro Restrepo a estudiar a fondo su papel durante el siglo XIX, la guerra sucia de las multinacionales en Colombia y las masacres en las zonas bananeras del país durante el siglo XX. (_Musa paradisíaca: Museo de Memoria de Colombia_, 2021).

Otro momento de crisis que aún hoy día sigue siendo atrayente de la mirada artística es el caso de La toma del Palacio de Justicia de Colombia durante noviembre de 1985, en ese evento murieron 98 personas y desaparecieron 11 más en medio del fuego cruzado y del incendio ocasionado durante la retoma militar. Este no fue un fenómeno social cualquiera, allí confluyeron circunstancias políticas, tácticas, de drama humano y manejo de medios de comunicación masiva. Al igual que en el caso del San Juan de Dios, la existencia de un gran entramado de atributos y dimensiones explicativas provoca una respuesta social, análisis sobre sistemas lógicos, sistemáticos y explicativos. En el caso del Palacio de Justicia inclusive se ha llegado a resoluciones con valor de verdad como la condena por la Corte Interamericana de Derechos Humanos.

Aún así persiste la reaparición de expresiones artísticas y acciones en memoria de lo ocurrido, la imaginación sigue retornando atributos de carácter explicativo que, a pesar de la prolífica literatura e investigación del fenómeno, éste, sigue “emanando” mensajes no-textuales y emocionales, en un flujo representativo que no agota nuestra flotante compresión de ese momento de crisis.

 

Ilustración 8. Señor presidente... Beatriz González (198).

**Señor presidente, qué honor estar con usted en este momento histórico** es una frase que le dijeron los ministros al presidente Belisario Betancur en el momento en que está sucediendo el incendio en el Palacio de Justicia.



Ilustración 9. Fragmentos del video Cajas Negras (2021)

Investigación y exposición visual sobre desaparición forzada en el Palacio de Justicia, proyecto encargado por la Comisión de la Verdad a Forensic Architecture.

En la ilustración #6 vemos un resultado visual de una experiencia de investigación transdisciplinar donde fueron aplicadas múltiples metodologías, que derivan en unos resultados visuales con una fuerte apuesta estética. _“Utilizamos nuestro modelo como dispositivo óptico con el que observamos nuevamente los videos de este evento, estas imágenes forman parte de la memoria colectiva colombiana, han sido reproducidas por décadas en las noticias, en documentales y en representaciones artísticas”,_ dice una voz en off en el video “Cajas negras” explicativo del proyecto de investigación publicado por _Forensic Architecture_.

En las nuevas imágenes creadas por _Forensic Architecture_ combinando técnicas de modelado virtual y superposición de grabaciones históricas, por más reciente y contemporánea que sea la imagen, en la experiencia de la mirada sobrevive la memoria social, en nosotros se reconfigura el pasado, el presente y el porvenir contenido en la imagen, se nos exige algo más que compresión, la mirada requiere nuestra humildad, humildad ante la imagen, y traspasar con la mirada la mera importancia histórica, liberarnos del pensador erudito, entregarse a la imagen, porque la imagen “quiere” algo, «la imagen no es un simple acontecimiento en el devenir histórico» (Didi-Huberman, 2011, p. 143). La imagen nos exige más de lo que explica, en ella deviene la paradoja del síntoma y el anacronismo. Por consiguiente, podemos esperar que, en las imágenes de registro a las respuestas críticas frente a una situación o evento de crisis, potencialmente existen esos síntomas y anacronismos.

## Producción artística y audiovisual HSJD

Durante el año 2007 María Elvira Escallón realizó una serie de instalaciones con el mobiliario del hospital, en donde se resalta la situación de abandono no solo de los artefactos sino también del espacio, estas obras surgen de un registro documental que comenzó en el año 2005, con este trabajo la artista ganó la convocatoria “Ciudad y patrimonio” del año 2006, obra editada por el Instituto Distrital de Patrimonio, publicación en la que escriben la curadora Natalia Gutiérrez y el arquitecto e historiador del HSJD David Cristancho.

Siguiendo la misma observación de la evidencia del paso del tiempo y la ruina en los espacios arquitectónicos del HSJD, pero con una mirada alrededor de sus habitantes, en 2011 se publica la obra fotográfica de Nicolás Van Hemelryck “+ San Juan Sin Dios”, premio Colombo Suizo de Fotografía.

Uno de los proyectos más interesantes considerando las categorías de análisis propuestas para esta investigación y la carga semántica en la imagen, es el guion para teatro _Tiempo de Dios_ de Juan Camilo Ahumada, quien desde el año 2001 a partir de unos testimonios y una entrevista a una trabajadora del HSJD empieza a producir esta pieza de dramaturgia que se publica en el año 2013, su obra relata la cotidianidad de los habitantes del San Juan y aún desconociendo las otras obras audiovisuales y artísticas escenifica las diversas formas de crisis y resistencia, reivindicando su lucha y respondiendo a la situación de desasosiego que generó el cierre del HSJD, ganador del premio Distrital de Dramaturgia de la Convocatoria Arte Dramático del Instituto Distrital de las Artes. Aunque sea una obra escrita, se puede afirmar que esta cargada de imágenes<sup>[\[7\]](#footnote-7)</sup> mentales con mucha carga simbólica, evocativa y emotiva.

En el año 2015 coincidieron una curaduría titulada _TimeBag Bogotá_ y el debate sobre la propiedad o responsabilidad estatal del HSJD, ese año la Superintendencia de Notariado y Registro confirmó que la entidad territorial que dispone de los predios del HSJD era la Gobernación de Cundinamarca. Y así, otras decisiones claves fueron tomadas en ese periodo de tiempo, como el destrabe del proceso de compra por parte del Distrito anunciado en varias ocasiones, y que se viera solo protocolariamente materializado en evento al que acudieron el alcalde Mayor de Bogotá de ese entonces, Gustavo Petro, el presidente de la República Juan Manuel Santos y el Gobernador de Cundinamarca, Álvaro Cruz.

En los ámbitos de intervención en el completo arquitectónico, se encuentra la intervención colectiva “Timebag Bogotá” donde los artistas Alejandro Arango, Alexandra Mccormick, Ana Karina Moreno, Fredy Alzate, Harold Ortiz, Jenniffer Duarte y Víctor Garcés exponen _in-situ en_ el HSJD para generar reflexión utilizando la homologación de conceptos médicos para señalar la crisis; un destacado espacio de reflexión que escenificó y presentó la oportunidad de llevar a la esfera pública la situación de abandono del Hospital y por supuesto las distintas pugnas de resistencia.

En otros enfoques y disciplinas encontramos la tesis de Pablo Jaramillo quien parte de la premisa de:

“lograr unificar estos dos predios que fueron desarticulados por la construcción de la carrera décima y generar una secuencia de recorridos, zonas de tránsito y permanencia a lo largo del proyecto para así lograr la articulación armoniosa entre cada una de las zonas propuestas y comenzar a vivir y recorrer el proyecto como un monumento más y devolverle al Conjunto Hospitalario lo que tanto lo caracterizaba en épocas anteriores: unión y belleza.” (Jaramillo U., 2016, p. 16).

Del mismo modo, encontramos la tesis de Janeth Marcela Rodríguez, quien también pretende asumir el San Juan de Dios como un lugar de gran riqueza arquitectónica, pero a su vez expresa gran preocupación porque este espacio sea usado para el bienestar de la ciudad, propone un proyecto de articulación entre lo arquitectónico, lo social y la salud del ciudadano como lo establece en su texto:

“El proyecto se enfoca en desarrollar y potencializar un espacio apropiado, dedicado a la investigación en temas de la salud, permitiendo la exploración tanto de entornos urbanísticos apropiados como de elementos arquitectónicos, dando paso a desarrollo médicos apropiados para PARQUE DE LA SALUD SAN JUAN DE DIOS el país y la región, generando una búsqueda constante en la implementación de métodos de investigación” (Neira et al., 2017, p. 8).

Otros textos se enfocaron en la necesidad de visibilizar la resistencia que han realizado las distintas personas que se negaron al cierre permanente del Hospital, señalando y buscando entender los procesos de resistencia que se han vivido dentro del Hospital.

“… Entender el fenómeno…implica grandes reflexiones sobre el rol de la ciudadanía, la salud en Colombia” (Orlando et al., 2015, p. 92)

El marco histórico y carácter emblemático del HSJD es usado como antesala para el desarrollo de complejas temáticas socioculturales, políticas e intentos de comprensión de la resistencia. La toma del hospital por parte de algunas personas transformó no solo el espacio sino también las vidas de quienes lo habitan transitoriamente. Hay muchas entrevistas a estos extrabajadores que dan cuenta de los hechos sucedidos, narrando de primera voz la resistencia.

Un informe multidisciplinar desarrollado entre las trabajadoras del San Juan de Dios y la Universidad Nacional evidencia la necesidad de hacer audible las voces de la protesta, demostrando que «las “comunidades” son todo menos inertes, pero que la manera en que “participan” en la vida política y exigen sus derechos no coincide necesariamente con la gramática y objetivos de la administración pública y de otros agentes gubernamentales, por el contrario, pareciera que en algunas situaciones, la “participación, social, comunitaria y ciudadana” se torna _incómoda»_ (Góngora et al., 2013, p. 5)_._

En dicho informe de corte etnográfico se tienen en cuenta el testimonio de los trabajadores del HSJD, es decir, se aparta de la historia oficial que generalmente es contada por los medios y “aparatos de dominación”, o en palabras de los autores:

“El objetivo de esta investigación es describir etnográficamente la lucha de las trabajadoras del HSJD por seguir siendo trabajadoras. Esto implica pensar en las posibilidades reales de tomar en serio el punto de vista de las “comunidades” que exigen sus derechos y pensar en verdades existenciales que se le escapan al idioma del derecho. Para esto, tendremos que evitar caer en la “denuncia”” (Góngora et al., 2013, p. 8).

Vemos que las investigaciones dan cuenta de las distintas situaciones que ocurren en simultáneo, dejan ver lo que hay más allá de la situación burocrática, abarcan la cotidianidad de los hombres y mujeres que hacen parte de este proceso y del hospital en sí.

Las piezas audiovisuales halladas son más de 50, entre argumentales, documentales y mucha reportería, cada uno de estos videos a su manera evidencia la situación por la que atravesaron las personas afectadas.

Se destaca el documental “_A la deriva”_ de Laura Díaz y Sara Trejos, realizado entre el año 2012 y 2013 construye una clara intención comunicativa en una narrativa lineal, observando la lucha de los trabajadores y la incertidumbre en la que se encontraba el Hospital para aquellos días.

Otra interesante pieza documental es “La Hortúa” de Andrés Chaves, una investigación sobre el HSJD que parte como un proyecto enfocado en lo fotográfico y testimonial pero que concluye en una propuesta estética sobre lo arquitectónico y vivencial. Muestra cómo ese hospital parecía la tierra después de los humanos, evidencia la desidia estatal visible en la ruina.

 Recorrido IDPC "La pulsión de la vida" exploración en el HSJD, 2022_