El Hospital San Juan de Dios se ubica en el centro geográfico del casco urbano de Bogotá, específicamente hacia los cerros surorientales en la localidad 15 Antonio Nariño, entre las calles 1 y 1a sur y las carreras 10 y 14 Avenida Caracas. El complejo hospitalario ocupa una superficie de 13 hectáreas que alberga 24 edificaciones, incluyendo el Hospital Materno Infantil, y contaba con 14 salas de cirugías y 500 camas que prestaron servicio hasta el año 2001. 

Vista a los cerros | Interior con jardines

Esta institución constituye un hito significativo en la historia de la resistencia social colombiana \parencite{Gongora2013}.

El sector se caracteriza por la presencia de importantes símbolos urbanísticos de relevancia histórica local y nacional. En su entorno inmediato se encuentran el parque Tercer Milenio, la iglesia del Voto Nacional y barrios tradicionales como el Policarpa, San Bernardo y Eduardo Santos. Asimismo, el área presenta una notable concentración de instituciones de salud, estableciendo vínculos con el Hospital Santa Clara, La Samaritana y el Hospital de la Misericordia, entre otros. La zona combina usos residenciales y comerciales, con predominio de la industria textil, mecánica y de fabricación de muebles y enseres.

La ubicación actual del hospital es resultado de un proceso histórico que se remonta al siglo XVI, cuando la Corona española ordenó la construcción de hospitales para atender tanto a nativos como a españoles. El primer hospital de Santafé, inicialmente denominado San Pedro, de Jesús, María y José, tuvo dos ubicaciones previas. La primera, como señala \parencite{Romero1994}, se estableció cuando \enquote{Fray Juan de los Barrios y Toledo otorgó escritura pública [...] donando unas casas de su propiedad situadas en la calle de San Felipe (hoy carrera sexta)}, exactamente detrás de la catedral primada, donde funcionaba con apenas doce camas. Posteriormente, en 1723, \enquote{con el producto de la venta de varias casas que pertenecían al Hospital, desde su primera fundación, y con buena parte de las limosnas recibidas, se inició la construcción de la nueva sede bajo la dirección de Fray Pedro Pablo de Villamar} en la calle San Miguel, actual ubicación entre carreras novena y décima con calles once y doce de Bogotá. Finalmente, la institución fue trasladada a los predios del \enquote{Molino de la Hortua}, una finca propiedad de José Domingo Ospina, donde permanece hasta la actualidad.

\begin{figure}[h]
\caption{Ubicación del HSJD}
\label{fig:ubicacion_hsjd}
\end{figure}


Cocina de tamaño medio, con una ventana grande que ocupa casi toda una pared. Iluminación: La luz natural entra por la ventana, creando sombras y contrastes en los objetos. Elementos Principales: Una persona de espaldas cocinando en una estufa. Una mesa grande cubierta de objetos, incluyendo platos sucios, bolsas, botellas y papeles. En las paredes hay cuadros y utensilios de cocina colgados. Detalles a considerar: El suelo está cubierto de objetos. La encimera está llena de utensilios de cocina. Hay una sensación general de desorden y acumulación de objetos. \parencite[fotograma: 00:34:25]{video_diaz2016}