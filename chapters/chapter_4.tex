La insistencia en observar el HSJD, es porque éste constituye un hito significativo en la historia de la resistencia social colombiana \parencite{Gongora2013}. Los objetos y espacios del hospital se han transformado en elementos icónicos que testimonian diferentes momentos históricos de la institución, desde instrumentos médicos abandonados hasta expresiones artísticas de protesta, constituyendo un archivo visual de su transformación y significado social.

El proceso de recolección de datos comenzó con una entrevista no estructurada a la enfermera Margarita Castro, quien proporcionó acceso a un valioso archivo audiovisual y documentación sobre las iniciativas de defensa del Hospital San Juan de Dios. La base documental se amplió mediante la búsqueda sistemática en fuentes secundarias, incluyendo registros de exposiciones artísticas, archivos de prensa, publicaciones académicas, documentales y material institucional. Posteriormente, en entrevista con el dramaturgo Juan Camilo Ahumada creador del guión ``Tiempo de dios'' se identificó que tanto en este gión como en otras activaciones en el HSJD, se puede visulmbrar un cierto patrón de estructura narrativa que tácitamente incluye imágenes-síntoma, montadas en estructuras tipo escenas o <<cuadros vivos>>\footnote{Como se menciona en el Apéndice (\ref{apendiceA}) \textit{...cada escena funciona de alguna manera de forma independiente, como un solo universo dentro de una lógica estructural que se propone como un recorrido. Cada una de estas escenas significa un espacio y  significa como una línea de acción que se cierra a sí misma.}}.

Los registros visuales bien sea de obra artística, reportería, argumental o documental potencialmente contienen imagen-síntoma y una inherente resistencia al olvido no por sí mismas, sino por aquello que vehiculan: son ventanas que permiten vislumbrar las ruinas como síntomas del drama humano y social del HSJD.

La base documental se amplió mediante la búsqueda sistemática en fuentes secundarias, incluyendo:

\begin{itemize}
\item Registros de exposiciones artísticas relacionadas con el HSJD
\item Archivos de prensa
\item Publicaciones académicas
\item Documentales y cortometrajes
\item Archivos de instituciones culturales y académicas
\item Noticias y reportajes televisivos
\end{itemize}

El repositorio final fue organizado en un repositorio, eliminando duplicados y estableciendo campos específicos para el análisis. El catálogo general comprende más de \textbf{800 registros} documentales como puede ver en el Apéndice (\ref{apendiceB}).


\section{El lugar, el objeto y situaciones icónicas}

Los objetos y espacios del hospital se han transformado en elementos icónicos que testimonian diferentes momentos históricos, desde instrumentos médicos abandonados hasta expresiones artísticas de protesta.

La transformación del Hospital San Juan de Dios (HSJD) y su significado social a través del tiempo constituye un hito fundamental en la historia de la resistencia social colombiana \parencite{Gongora2013}. La lucha por su preservación representa no solo una batalla por la infraestructura física, sino por la memoria colectiva y el derecho a la salud pública. Los trabajadores que permanecieron en el hospital después de su cierre transformaron los espacios hospitalarios en símbolos de resistencia, evidenciando la tensión entre el abandono institucional y la persistencia de la comunidad hospitalaria.

El análisis de las representaciones visuales del HSJD revela tres momentos significativos de una misma imagen-síntoma. En primer lugar, encontramos una imagen hallada en la sala de espera del Pabellón de Consulta por la artista Luisa Fernanda Vela, acompañada de su autorretrato con la inscripción \enquote{Si no hay justicia, hay performance}. El segundo momento corresponde al registro del performance \enquote{Hortua in-hospitalario} de David Lozano, donde aparece Luisa Margarita Castro Gonzáles, enfermera e investigadora, cuyo activismo ejemplifica la resistencia ante la crisis del hospital. El tercer momento muestra nuevamente a Margarita en 2022, realizando el recorrido titulado \enquote{Siga esta es su casa}, vistiendo el uniforme y toga tradicional de enfermería.

En el análisis de las manifestaciones visuales relacionadas con el Hospital San Juan de Dios (HSJD), se evidencian múltiples diálogos simbólicos que emergen de las representaciones del complejo hospitalario. La obra "En estado de coma" de Elvira Escallón constituye una intervención \textit{in situ} que, mediante una economía de elementos plásticos, logra una imagen poética que manifiesta síntomas y anacronismos. La artista traslada lo biológico que germina en la ruina arquitectónica hacia el objeto instrumental: las camas sobre las que crece el césped insinúan la presencia del sujeto, sugiriendo la aparición del yo en el espacio hospitalario.

La intervención "Hortua in-hospitalario" de David Lozano presenta un segundo nivel de análisis, donde un actor ejecuta acciones corporales que resignifican el objeto. Esta composición, también construida a partir de objetos encontrados, proyecta signos denotativos de la crisis del San Juan. Por su parte, el documental "La Hortúa" de Andrés Cháves ofrece un tercer encuadre significativo. El contexto de producción, marcado por la ocupación informal del predio, obligó a un registro cauteloso con cámara en mano. Un fotograma particular captura una camilla en abandono, metáfora potente del estado de la salud pública y la crisis institucional.

Los recorridos de sensibilización realizados en 2022 por el Instituto Distrital de Patrimonio Cultural (IDPC) constituyen un hito de análisis. Es notable cómo estas <<puestas en escena>>, aun en ausencia del objeto-cama, mantienen la potencia simbólica a través del contexto de uso y la acción performativa, replicando los signos icónicos presentes en las otras manifestaciones artísticas.

Los signos denotativos, anacronismos e imágenes-síntoma mantienen su capacidad discursiva mientras conserven algún elemento simbólico de las acciones de crisis y resistencia, sin requerir referencias explícitas al contexto histórico-social. Estas imágenes, analizadas desde el concepto de imagen-síntoma, contienen una carga de resistencia al olvido, no en sí mismas, sino en aquello que vehiculan o posibilitan: son ventanas que permiten vislumbrar ruinas que actúan como síntomas del drama humano y social.

\section{Yuxtaposición hogar-hospital}

La arquitectura del hospital refleja una dualidad entre lo institucional y lo doméstico, donde los espacios fueron apropiados por sus habitantes como lugares de vida y trabajo simultáneamente. Esta yuxtaposición entre hogar y hospital se materializa en los jardines interiores, las zonas comunes y las adaptaciones espaciales realizadas por el personal médico y administrativo que habitó el complejo.

La transformación del HSJD constituye un hito en la historia de la resistencia social colombiana \parencite{Gongora2013}. Los trabajadores que permanecieron después del cierre transformaron los espacios en símbolos de resistencia, evidenciando la tensión entre el abandono institucional y la persistencia de la comunidad hospitalaria.

Las imágenes actúan como contenedoras de las modalidades del tiempo, albergando frágiles supervivencias que provocan emociones y comprensiones no-verbales. En el contraste entre hogar y hospital se materializan motivos iconográficos, síntomas y anacronismos, extrañas conjunciones entre diferencia y repetición. Cuando las imágenes ``dialogan'' con las categorías de análisis, también exhibe una complejidad inherente que dificulta la clasificación y organización de las relaciones emergentes. La memoria social y la imagen artística establecen una relación bidireccional, donde las redes significantes no necesariamente siguen una jerarquía, sistema lógico o linealidad narrativa.