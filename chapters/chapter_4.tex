El proceso de recolección de datos comenzó con una entrevista no estructurada a la enfermera Margarita Castro, quien proporcionó acceso a un valioso archivo audiovisual y documentación sobre las iniciativas de defensa del Hospital San Juan de Dios, incluyendo el proyecto de recorridos culturales ``Siga esta es su casa'', archivos de presentaciónes en formato Power Point preparados por ella para distintas audiencias y registros fotográficos de lo que podría definirse como testimonios de circunstancias en el devenir de los acontecimientos en el HSJD.
 

La base documental se amplió mediante la búsqueda sistemática en fuentes secundarias, incluyendo:
\begin{itemize}
\item Registros de exposiciones artísticas relacionadas con el HSJD
\item Archivos de prensa
\item Publicaciones académicas
\item Documentales y cortometrajes
\item Archivos de instituciones culturales y académicas
\item Noticias y reportajes televisivos
\end{itemize}

Adicionalmente, se realizaron entrevistas no estructuradas a los artistas David Lozano y Juan Camilo Ahumada, quienes compartieron material visual y textual inédito hasta ese momento.

El repositorio final fue organizado en una base de datos relacional, eliminando duplicados y estableciendo campos específicos para el análisis. El catálogo general comprende \textbf{1048 registros} documentales (véase anexo ``Base de datos imagen HSJD'').

Esta institución constituye un hito significativo en la historia de la resistencia social colombiana \parencite{Gongora2013}. Los objetos y espacios del hospital se han transformado en elementos icónicos que testimonian diferentes momentos históricos de la institución, desde instrumentos médicos abandonados hasta expresiones artísticas de protesta, constituyendo un archivo visual de su transformación y significado social.

% Realizaré la corrección por partes dado lo extenso del texto. Me enfocaré en mejorar la estructura, coherencia y formato LaTeX manteniendo el contenido conceptual.

La transformación del Hospital San Juan de Dios (HSJD) y su significado social a través del tiempo constituye un hito fundamental en la historia de la resistencia social colombiana \parencite{Gongora2013}. La lucha por su preservación representa no solo una batalla por la infraestructura física, sino por la memoria colectiva y el derecho a la salud pública. Los trabajadores que permanecieron en el hospital después de su cierre transformaron los espacios hospitalarios en símbolos de resistencia, evidenciando la tensión entre el abandono institucional y la persistencia de la comunidad hospitalaria.

\section{Objetos y situaciones icónicas}

El sector donde se ubica el HSJD se caracteriza por la presencia de importantes símbolos urbanísticos de relevancia histórica local y nacional. En su entorno inmediato se encuentran el parque Tercer Milenio, la iglesia del Voto Nacional y barrios tradicionales como el Policarpa, San Bernardo y Eduardo Santos. El área presenta una notable concentración de instituciones de salud, estableciendo vínculos con el Hospital Santa Clara, La Samaritana y el Hospital de la Misericordia, entre otros. La zona combina usos residenciales y comerciales, con predominio de la industria textil, mecánica y de fabricación de muebles y enseres.

La ubicación actual del hospital es resultado de un proceso histórico que se remonta al siglo XVI, cuando la Corona española ordenó la construcción de hospitales para atender tanto a nativos como a españoles. El primer hospital de Santafé, inicialmente denominado San Pedro, de Jesús, María y José, tuvo dos ubicaciones previas. La primera, como señala \parencite{Romero1994}, se estableció cuando \enquote{Fray Juan de los Barrios y Toledo otorgó escritura pública [...] donando unas casas de su propiedad situadas en la calle de San Felipe (hoy carrera sexta)}, exactamente detrás de la catedral primada, donde funcionaba con apenas doce camas.

% Procederé a corregir el texto mejorando su estructura y estilo, manteniendo el formato LaTeX.

La evolución histórica del Hospital San Juan de Dios (HSJD) refleja una serie de transformaciones espaciales que han marcado su desarrollo institucional. Inicialmente denominado Hospital San Pedro, de Jesús, María y José, la institución atravesó por tres ubicaciones distintas que definieron su trayectoria. Como señala \textcite{Romero1994}, su primera sede se estableció cuando \enquote{Fray Juan de los Barrios y Toledo otorgó escritura pública [...] donando unas casas de su propiedad situadas en la calle de San Felipe} (actual carrera sexta), detrás de la catedral primada, donde operaba modestamente con doce camas.

Posteriormente, en 1723, gracias a recursos provenientes de la venta de propiedades originales y donaciones recibidas, se inició la construcción de una nueva sede bajo la dirección de Fray Pedro Pablo de Villamar, ubicada en la calle San Miguel (actual intersección entre carreras novena y décima con calles once y doce de Bogotá). Finalmente, el hospital fue trasladado a los predios del \enquote{Molino de la Hortua}, una propiedad de José Domingo Ospina, donde permanece hasta la actualidad.

El análisis de las representaciones visuales del HSJD revela tres momentos significativos de una misma imagen-síntoma. En primer lugar, encontramos una imagen hallada en la sala de espera del Pabellón de Consulta por la artista Luisa Fernanda Vela, acompañada de su autorretrato con la inscripción \enquote{Si no hay justicia, hay performance}. El segundo momento corresponde al registro del performance \enquote{Hortua in-hospitalario} de David Lozano, donde aparece Luisa Margarita Castro Gonzáles, enfermera e investigadora, cuyo activismo ejemplifica la resistencia ante la crisis del hospital. El tercer momento muestra nuevamente a Margarita en 2022, realizando el recorrido titulado \enquote{Siga esta es su casa}, vistiendo el uniforme y toga tradicional de enfermería.

% Procederé a corregir el texto mejorando su estructura y estilo académico, manteniendo el formato LaTeX.

En el análisis de las manifestaciones visuales relacionadas con el Hospital San Juan de Dios (HSJD), se evidencian múltiples diálogos simbólicos que emergen de las representaciones del complejo hospitalario. La obra "En estado de coma" de Elvira Escallón constituye una intervención \textit{in situ} que, mediante una economía de elementos plásticos, logra una imagen poética que manifiesta síntomas y anacronismos. La artista traslada lo biológico que germina en la ruina arquitectónica hacia el objeto instrumental: las camas sobre las que crece el césped insinúan la presencia del sujeto, sugiriendo la aparición del yo en el espacio hospitalario.

La intervención "Hortua in-hospitalario" de David Lozano presenta un segundo nivel de análisis, donde un modelo ejecuta acciones corporales que resignifican el objeto. Esta composición, también construida a partir de objetos encontrados, proyecta signos denotativos de la crisis del San Juan. Por su parte, el documental "La Hortúa" de Andrés Cháves ofrece un tercer encuadre significativo. El contexto de producción, marcado por la ocupación informal del predio, obligó a un registro cauteloso con cámara en mano. Un fotograma particular captura una camilla en abandono, metáfora potente del estado de la salud pública y la crisis institucional.

Los recorridos de sensibilización realizados en 2022 por el Instituto Distrital de Patrimonio Cultural (IDPC) constituyen un cuarto elemento de análisis. Es notable cómo estas puestas en escena, aun en ausencia del objeto-cama, mantienen la potencia simbólica a través del contexto de uso y la acción performativa, replicando los signos icónicos presentes en las otras manifestaciones artísticas.

Los signos denotativos, anacronismos e imágenes-síntoma mantienen su capacidad discursiva mientras conserven algún elemento simbólico de las acciones de crisis y resistencia, sin requerir referencias explícitas al contexto histórico-social. Estas imágenes, analizadas desde el concepto de imagen-síntoma, contienen una carga de resistencia al olvido, no en sí mismas, sino en aquello que vehiculan o posibilitan: son ventanas que permiten vislumbrar ruinas que actúan como síntomas del drama humano y social.

% Procederé a corregir el texto mejorando su estructura y estilo, manteniendo el contenido conceptual.

Las imágenes analizadas bajo el concepto de imagen-síntoma poseen una inherente resistencia al olvido, no por sí mismas, sino por aquello que contienen y permiten visualizar. Son imágenes que posibilitan vislumbrar ruinas que actúan como síntomas del drama humano y social.

\section{Yuxtaposición hogar-hospital}

La ubicación actual del hospital es resultado de un proceso histórico que se remonta al siglo XVI, cuando la Corona española decretó la construcción de hospitales para atender tanto a nativos como a españoles. El primer hospital de Santafé, inicialmente denominado San Pedro, de Jesús, María y José, tuvo dos ubicaciones previas. La primera, como señala \textcite{Romero1994}, se estableció cuando \enquote{Fray Juan de los Barrios y Toledo otorgó escritura pública [...] donando unas casas de su propiedad situadas en la calle de San Felipe} (actual carrera sexta), exactamente detrás de la catedral primada, donde funcionaba con apenas doce camas. 

Posteriormente, en 1723, \enquote{con el producto de la venta de varias casas que pertenecían al Hospital, desde su primera fundación, y con buena parte de las limosnas recibidas, se inició la construcción de la nueva sede bajo la dirección de Fray Pedro Pablo de Villamar} en la calle San Miguel, actual ubicación entre carreras novena y décima con calles once y doce de Bogotá. Finalmente, la institución fue trasladada a los predios del \enquote{Molino de la Hortua}, una finca propiedad de José Domingo Ospina, donde permanece hasta la actualidad.

La arquitectura del hospital refleja una dualidad entre lo institucional y lo doméstico, donde los espacios fueron apropiados por sus habitantes como lugares de vida y trabajo simultáneamente. Esta yuxtaposición entre hogar y hospital se materializa en los jardines interiores, las zonas comunes y las adaptaciones espaciales realizadas por el personal médico y administrativo que habitó el complejo.

% Procederé a mejorar el estilo del texto, organizándolo en secciones más claras y mejorando su fluidez. También añadiré los comandos LaTeX necesarios para las ilustraciones.

\section{Análisis visual y propuesta de experiencia inmersiva}

\subsection{Análisis de montajes visuales}

Los montajes analizados revelan patrones significativos en la representación visual de la crisis del Hospital San Juan de Dios (HSJD):

\begin{figure}[h]
\caption{Montaje e1m1}
\begin{itemize}
\item \textbf{Anacronismo:} El vaivén temporal de la ocupación y la urgencia de habitar a destiempo
\item \textbf{Imagen-síntoma:} Manifestaciones corporales a través de la piel y la vestimenta
\end{itemize}
\end{figure}

\begin{figure}[h]
\caption{Montaje e2m1}
\begin{itemize}
\item \textbf{Anacronismo:} La paradoja temporal entre auxilio y abandono
\item \textbf{Imagen-síntoma:} La dualidad entre el sonido latente de la sirena y el silencio imperante
\end{itemize}
\end{figure}

\begin{figure}[h]
\caption{Montaje e3m1}
\begin{itemize}
\item \textbf{Anacronismo:} El contradiscurso del impulso vital en momentos de crisis
\item \textbf{Imagen-síntoma:} Manifestaciones gestuales del cuidado de la vida
\end{itemize}
\end{figure}

\subsection{Propuesta de experiencia inmersiva}

Se propone desarrollar una experiencia de inmersión en ambiente virtual (VR) como espacio para la memoria y comunicación de las observaciones artísticas sobre el HSJD. La elección de la realidad virtual responde a la necesidad de enfatizar las idealizaciones, imaginarios y superposiciones de signos icónicos, imágenes-síntoma y anacronismos identificados en la investigación.

% Reorganizaré el texto para mejorar su coherencia y fluidez, manteniendo el contenido conceptual pero mejorando su estructura.

Esta investigación propone una experiencia inmersiva que vincula imagen y memoria social. El Hospital San Juan de Dios (HSJD), desde el inicio de su proceso de liquidación, ha permanecido cerrado, exceptuando algunos eventos específicos relacionados con la memoria, el patrimonio y el activismo social. Considerando su indudable valor patrimonial, potencia poética y su inminente transformación, se propone un diseño de experiencia que incorpora simulaciones inmersivas con énfasis en la sensibilización sensorial y perceptiva.

La propuesta se fundamenta en los hallazgos previos donde la escenificación ha sido un mecanismo recurrente de activismo y expresión. Se apela a la capacidad de las personas para ordenar con la mirada y crear imaginarios, buscando conmemorar las acciones de reconocimiento del drama humano vivido en el HSJD, las luchas sociales y potenciar las experiencias poéticas y estéticas que han convergido alrededor de esta crisis y sus resistencias.

\textbf{Características del diseño}

El usuario asume el rol de personaje en primera persona, explorando un espacio donde surgen dramatizaciones a través de portales que lo introducen en cuadros vivos. Este patrón se repite, conduciendo al usuario por historias no-textuales que, concatenadas, permiten participar de una experiencia estética, memorial e informativa. La característica esencial es la coexistencia de la representación arquitectónica del complejo hospitalario con la simulación de escenarios en formato de cuadros vivos.

\textbf{Género y mecánica}

Se clasifica como un juego serio o juego para el cambio social, diseñado con propósito informativo y cultural. La mecánica principal se basa en portales ubicados en la superficie arquitectónica del HSJD, que funcionan como umbrales bidireccionales entre el ambiente ``realista'' de exploración y las escenas que recrean poéticamente las experiencias de los síntomas dialógicos y anacronismos.

% Procederé a mejorar la estructura y estilo del texto, manteniendo el contenido conceptual pero haciéndolo más académico y fluido.

La investigación desarrolla un análisis de la memoria visual y artística relacionada con el Hospital San Juan de Dios (HSJD) de Bogotá durante su período de crisis. El estudio se materializa a través de una propuesta de juego serio, diseñado con propósitos informativos y culturales, que emplea la simulación mientras enfatiza valores pedagógicos, memoriales y estéticos.

La mecánica principal del juego se basa en un sistema de portales, ubicados estratégicamente en la representación arquitectónica del HSJD. Estos portales, que funcionan en pares bidireccionales, permiten al usuario transitar entre dos tipos de espacios: un ambiente ``realista'' de exploración en mundo abierto y escenas que recrean poéticamente experiencias de síntomas dialógicos y anacronismos, facilitando recorridos interpretativos mediante la mirada e interacción del usuario.

El análisis del corpus visual reveló que la selección misma de las imágenes constituye parte fundamental del proceso analítico. Si bien la interpretación de imágenes puede resultar abrumadora, existe un sólido marco teórico-metodológico para abordar el análisis visual en contextos de problemáticas sociales, políticas y culturales. Esta investigación se fundamentó principalmente en la antropología de la imagen y métodos de ``investigación sensible''.

Las imágenes, como insumo investigativo, permiten explorar la memoria social y el inconsciente colectivo, revelando aspectos que pueden pasar inadvertidos en fuentes textuales. La creación de sentido requiere capacidad para identificar signos dialógicos, lo cual demanda una profunda inmersión en el tema y habilidad para reconocer síntomas, latencias y contextos de producción visual.

% He reorganizado y mejorado el texto manteniendo las ideas principales pero mejorando su fluidez y estructura. He agregado referencias en formato APA donde era necesario.

La construcción de sentido a través del análisis de imágenes requiere una profunda inmersión en el tema y la capacidad de identificar signos dialógicos, síntomas, latencias y contextos de producción. En este proceso, la mirada analítica juega un papel fundamental al ordenar y dar sentido, siendo capaz de percibir las disrupciones visuales que irrumpen en la cotidianidad. Como señala \parencite{Didi-Huberman2011}, las imágenes no son elementos neutrales, sino que contienen deseos y estimulan la imaginación, materializando complejos entramados dialógicos y emocionales.

Las imágenes actúan como contenedoras de las modalidades del tiempo, albergando frágiles supervivencias que provocan emociones y comprensiones no-verbales. La investigación a través de la imagen revela aspectos del fenómeno materializados en motivos iconográficos, síntomas y anacronismos. Según \parencite{Warburg2010}, la imagen-síntoma representa una extraña conjunción entre diferencia y repetición, una irrupción en el curso normal de la representación que el investigador social puede aprovechar mediante el adiestramiento de su mirada y semántica visual.

Para el análisis de imágenes derivadas de fenómenos sociocomunicativos no-domesticados, es decir, aquellos que contienen expansiones de sentido en redes dialógicas no-textuales, se empleó el método del "montaje". Este enfoque, desarrollado por \parencite{Benjamin1999}, propicia la emergencia de la imagen-síntoma y los indicios de anacronismos, permitiendo al investigador "jugar" con las imágenes como estrategia para abordar ideas abstractas, asociarlas y organizarlas espacialmente.

% He reorganizado y mejorado el texto manteniendo las ideas principales pero mejorando su estructura y fluidez. También he agregado algunas transiciones para mejorar la coherencia.

La construcción de sentido en torno a problemáticas sociales complejas requiere un análisis profundo de sus representaciones visuales. Si bien la imagen demanda más de lo que explica explícitamente, su análisis sistemático permite expandir nuestra comprensión de fenómenos sociales complejos. Aunque la imagen puede resistirse a una interpretación unívoca, posee atributos que, mediante una adecuada articulación teórico-metodológica, facilitan la construcción de significado.

La investigación basada en imágenes presenta una naturaleza paradójica: mientras resulta reveladora cuando las imágenes ``dialogan'' con las categorías de análisis, también exhibe una complejidad inherente que dificulta la clasificación y organización de las relaciones emergentes. La memoria social y la imagen artística establecen una relación bidireccional, donde las redes significantes no necesariamente siguen una jerarquía, sistema lógico o linealidad narrativa.

En este contexto, la imagen ---entendida como la expresión material y visual de la obra de arte y los archivos fotográficos--- puede constituirse como sujeto de análisis. Sin embargo, siguiendo la práctica conceptual del montaje propuesta por \parencite{warburg1999mnemosyne}, nos situamos desde una mirada parcial donde la construcción de sentido se completa a través de la experiencia del espectador.

% El texto continúa, indico

% He mejorado la estructura, coherencia y estilo académico del texto, manteniendo su contenido conceptual.

El análisis del periodo estudiado revela un discurso oscilante entre los anuncios de apertura y cierre del Hospital San Juan de Dios (HSJD). Este proceso culminó en el desalojo forzado del complejo hospitalario, mientras que la prometida reapertura nunca se materializó, a pesar de contar con el respaldo de importantes figuras políticas como gobernadores, alcaldes y presidentes.

Al examinar la producción de imágenes culturales, artísticas y memoriales relacionadas con la crisis del HSJD, se observan patrones recurrentes en los gestos de enunciación. Estas similitudes, reveladas mediante el montaje, evidencian relaciones significativas entre la mirada del intérprete y ciertos atributos del fenómeno que funcionan como atractores visuales.

El montaje permite identificar repeticiones en la mirada que trascienden la mera centralidad del espacio físico y arquitectónico del complejo hospitalario de La Hortúa. Estos atractores visuales pueden categorizarse como signos denotativos y organizarse en escenas o montajes, facilitando una interpretación no textual de las redes significantes e imaginarias latentes en cada conjunto de imágenes.

La construcción final de sentido recae en el observador o nuevo espectador, a quien se le presentan estas escenas y se le proporciona la información necesaria para sumergirse en el discurso de crisis del HSJD. Los distintos montajes, caracterizados por sus síntomas dialógicos y anacronismos, conforman senderos interpretativos que posibilitan la construcción de significado.

% Se sugiere incluir una referencia a la imagen mencionada al final del texto original ("Interior Edificio Central HSJD. Cartel señalética Sala de Consulta Externa") en un pie de figura utilizando el comando \caption{} de LaTeX.