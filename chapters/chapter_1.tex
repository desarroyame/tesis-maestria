\section*{Introducción al fenómeno}

La crisis del Hospital San Juan de Dios (HSJD) emerge como un laboratorio social donde el diseño y la creación interactiva permiten explorar las dinámicas de transformación institucional, memoria colectiva y resistencia ciudadana. Este estudio trasciende la mera indagación documental o artística para analizar cómo diversas acciones simbólicas, argumentales y estéticas evidencian una intensa producción visual en torno a la crisis del HSJD. Los registros de estas acciones conforman un corpus de imágenes que permite interpretar tanto los síntomas visuales de la crisis sistémica como el deterioro arquitectónico y patrimonial en su interacción con el drama humano y social.

El caso del HSJD resulta emblemático en el contexto de las crisis institucionales generadas por las transformaciones masivas en los sistemas de salud pública. Surge entonces la pregunta: ¿cómo comprender, explicar e interpretar los registros visuales y las obras que actúan como evidencias del drama humano y la defensa de un bien público?

Estas evidencias, que denominamos imagen-síntoma, constituyen presencias disruptivas que interrumpen el curso normal de la representación: supervivencias, latencias y reapariciones que habitan las imágenes. La metodología aplicada a esta colección visual de memoria social permite abordar el fenómeno a través de la imagen.

El concepto de imagen-síntoma\footnote{\begin{quote}La idea de la imagen como síntoma, tomada por Didi-Huberman de Freud y aplicada al campo de la historia del arte, reconoce las diferencias entre disciplinas y experiencias. Sin embargo, el uso del carácter sintomático de la imagen en la historia del arte mantiene paralelismos con el análisis de los sueños, aunque aquí se utiliza para nombrar la perturbación que lo visual causa dentro de lo visible \parencite[p. 37]{VegaArevalo2017}\end{quote}} resulta fundamental para comprender cómo las representaciones visuales del HSJD trascienden la mera documentación para convertirse en manifestaciones de tensiones sociales más profundas. Como señala Didi-Huberman:

\begin{quote}
Si la imagen es un síntoma -en el sentido crítico y no clínico del término-, si la imagen es un malestar en la representación, es porque indica un futuro de la representación, un futuro que no sabemos aún leer, ni, incluso, describir \parencite[p. 177]{DidiHuberman2011}.
\end{quote}

Este estudio profundiza en el papel fundamental de la imagen para la contemplación de los fenómenos sociales, ofreciendo nuevas perspectivas para comprender su complejidad y contribuyendo al debate sobre el rol del arte y el diseño en los procesos de cambio social y la construcción de memoria colectiva.

El diseño, entendido como práctica crítica y performativa, permite develar las capas de significado ocultas en los registros visuales. No se trata únicamente de documentar, sino de construir narrativas que revelen las tensiones sociales subyacentes. Las imágenes del HSJD se transforman así en interfaces de memoria, donde cada fragmento, ruina y registro artístico funciona como un nodo de información compleja.

La creación interactiva se presenta como una metodología para trascender la observación pasiva, promoviendo una participación activa en la construcción de sentido. En este contexto, las obras artísticas y los registros visuales funcionan como dispositivos de activación memorial que permiten:

\begin{itemize}
    \item Desarticular narrativas oficiales sobre el abandono institucional
    \item Visibilizar las experiencias de trabajadores y comunidades afectadas
    \item Generar nuevas formas de comprensión y elaboración del conflicto social
\end{itemize}

\section*{Contextualización histórica}

El Hospital San Juan de Dios (HSJD) de Bogotá, fundado en 1723, se estableció como un centro pionero de atención médica y asistencia social para la población vulnerable. Durante casi tres siglos, esta institución se consolidó como un referente fundamental de la salud pública en Colombia, destacándose tanto por su labor asistencial como por su rol en la formación de profesionales de la salud. Sin embargo, la década de 1990 marcó el inicio de una serie de crisis institucionales y financieras que culminarían en su cierre.

El año 2001 representa un punto de inflexión en la historia del hospital, cuando "3.640 personas quedaron desempleadas, y aproximadamente 1.500 trabajadores no recibieron la liquidación correspondiente por sus años de servicio" \parencite{Castiblanco2017}. Este acontecimiento catalizó múltiples formas de resistencia ciudadana que trascendieron la mera reivindicación laboral, evidenciando un complejo entramado de problemáticas sociales.

La crisis del HSJD tiene raíces profundas que anteceden al cese de sus operaciones. Según el investigador Mario Hernández, especialista en historia de la medicina, las décadas de 1970 y 1980 marcaron el inicio de la crisis de los Estados de Bienestar, período caracterizado por la implementación de políticas neoliberales. La presión por incorporar los servicios de salud a las dinámicas de mercado impulsó una transformación institucional que el hospital, originalmente concebido bajo un modelo de beneficencia, no logró asimilar exitosamente. Esta transición fallida se manifestó en el deterioro progresivo de su infraestructura patrimonial, aunque el hospital mantuvo su función asistencial hasta su último día de operación.

Paralelamente a las iniciativas institucionales para su reapertura, emergieron diversas expresiones de resistencia social, manifestaciones artísticas y acciones conmemorativas que abordaron la complejidad del fenómeno. Esta investigación se centra en el análisis de la construcción de sentido a través de obras e imágenes artísticas, estéticas y poéticas no funcionales.

\subsubsection*{Exhibición y creación}

El HSJD se ha convertido en un punto focal para diversos campos disciplinares, atrayendo la atención de urbanistas, antropólogos y artistas plásticos, quienes encuentran en sus espacios un rico territorio para la reflexión y la creación. Este caso paradigmático ha catalizado el debate público sobre la salud como derecho fundamental, manifestándose a través de diversas prácticas disciplinares y expresiones simbólicas.

A partir de 2007, se observa una proliferación sistemática de manifestaciones artísticas que abordan explícitamente la problemática del HSJD. Desde entonces, se han producido y exhibido con regularidad casi anual obras artísticas, intervenciones estéticas y ocupaciones arquitectónicas relacionadas con esta situación.

\begin{table}[h!]
\centering
\begin{tabular}{|l|l|l|}
\hline
\textbf{Año} & \textbf{Artista} & \textbf{Obra} \\ \hline
2007 & María Elvira Escallón & En estado de coma \\ \hline
2011 & Nicolas Van Hemelryck & San Juan sin Dios \\ \hline
2013 & Juan Camilo Ahumada & Tiempo de dios (guión para teatro) \\ \hline
2015 & Fredy Alzate & Quiste \\ \hline
2015 & Alexandra Mccormick & Potenciales Evocados para aplicaciones clínicas \\ \hline
2015 & Víctor Garcés & Juan N de Dios \\ \hline
2015 & Jenniffer Duarte & Didácticos para una sala de espera \\ \hline
2015 & Ana Karina Moreno & Una más de las resistencias \\ \hline
2015 & Nathaly Rubio & Lo mejor es que nos olvidamos \\ \hline
2015 & Harold Ortiz & Sala de espera \\ \hline
2015 & Alejandro Arango & Egotherapy \\ \hline
2016 & David Lozano & Hortua inhospitalario \\ \hline
2017 & Luisa Fernanda Vela & Al margen \\ \hline
\end{tabular}
\caption{Obras y artistas relacionados con el HSJD, 2007-2017.}
\label{tabla:obras_artistas}
\end{table}

Complementando estas obras, se han realizado diversas activaciones in situ, incluyendo el \textit{Concierto Todo para la Paz} y \textit{Time Bag Bogotá} (2015). Esta tendencia ha continuado más allá del marco temporal de este estudio, con eventos como la \textit{Feria del Millón} (2021), \textit{Siga esta es su casa} (2022) y \textit{La pulsión de la vida} - Activaciones sociales del IDPC.

\section*{Planteamiento del problema}

La crisis del Hospital San Juan de Dios (HSJD) de Bogotá plantea interrogantes fundamentales sobre la respuesta social ante el abandono institucional: ¿Qué acciones tomar? ¿Permanecer, partir, resistir? ¿Por qué y hasta cuándo? Si bien el egreso del último paciente en 2001 marcó un hito significativo, este evento no representó el cierre definitivo del hospital. El proceso de deterioro institucional había iniciado años antes, y las gestiones para su liquidación y eventual reapertura se han extendido por décadas, situando al HSJD en un peculiar estado de latencia activa.

Paralelamente a los esfuerzos institucionales, el HSJD ha sido escenario de diversas manifestaciones de resistencia social, investigación académica y expresión artística que abordan la complejidad del fenómeno. Mientras el hospital experimentaba su desmaterialización funcional, emergían simultáneamente acciones de construcción de sentido en sus dimensiones patrimoniales, estéticas y poéticas. Este contexto suscita interrogantes cruciales sobre la naturaleza de la imagen artística y su capacidad para expresar la crisis y resistencia en el HSJD.

Para analizar los registros gráficos y audiovisuales —que incluyen procesos de creación artística, instalaciones \textit{in situ}, performances y una obra de dramaturgia— se adopta el concepto de \textit{imagen-síntoma}. Este enfoque trasciende el análisis formal de las imágenes para examinar su capacidad de revelar manifestaciones del drama humano y social, considerando que estos registros contienen signos icónicos e indicios de la imaginación social colectiva.

El corpus de imágenes relacionadas con el HSJD exhibe recurrencias discursivas provenientes de diversos ámbitos de participación, evidenciando tanto la crisis como las respuestas críticas ante situaciones que generan más interrogantes que certezas. Estas representaciones visuales constituyen un discurso que demanda una interpretación desde una perspectiva crítica y multidimensional.

En este contexto, surge la pregunta central: ¿De qué manera las representaciones visuales y artísticas de la crisis del HSJD contribuyen a la construcción de la memoria colectiva y al entendimiento de problemáticas sociales sistémicas? ¿Cómo pueden estas representaciones ser reinterpretadas, mediante el montaje y la experiencia visual, para generar narrativas transformadoras que estimulen el debate social?

De esta interrogante principal se desprenden tres preguntas orientadoras:

\begin{enumerate}
    \item \textbf{¿Cómo se representa la crisis y la resistencia en el HSJD a través del arte y la memoria visual?}  
    Esta pregunta examina la categoría de representación social y memoria, analizando el marco histórico-contextual y el rol del arte en la resistencia y preservación de la memoria colectiva.

    \item \textbf{¿Qué significado adquiere la imagen artística en el contexto de crisis social y resistencia?}  
    Vinculada al pensamiento visual y la construcción de sentido, esta pregunta explora cómo las imágenes artísticas facilitan la comprensión de situaciones sociales complejas, considerando su dimensión temporal y contextual.

    \item \textbf{¿Cómo las obras visuales sobre el HSJD establecen una relación simbólica entre el entorno social y el espectador?}  
    Esta interrogante, relacionada con el montaje y la experiencia visual, analiza las conexiones simbólicas y el potencial de las imágenes para comunicar experiencias, generar significados y promover la reflexión sobre la crisis y la resistencia.
\end{enumerate}

\section*{Justificación y relevancia}

La crisis del Hospital San Juan de Dios (HSJD) constituye una problemática sistémica donde convergen múltiples dimensiones interconectadas: económicas, administrativas, sociales, políticas y culturales. Desde la perspectiva del diseño y la creación interactiva, resulta pertinente abordar estas problemáticas sociales complejas mediante enfoques y metodologías innovadoras, especialmente en el actual contexto colombiano de construcción de paz. En este escenario, la transformación social requiere trascender la mera resolución del conflicto para atender otros ámbitos del bienestar social \parencite[p. 313]{Capra1998}.

El presente estudio se fundamenta en un corpus documental que comprende más de 1.000 registros visuales, cuidadosamente seleccionados de producciones académicas y artísticas que abordan la crisis y cierre del HSJD. Resulta significativo observar la correlación temporal entre los momentos más álgidos de la crisis hospitalaria y el incremento en la producción artística, fenómeno que coincide además con una intensificación en la cobertura mediática del caso por parte de los principales medios de comunicación bogotanos.

Este fenómeno visual-documental demuestra cómo la crisis del HSJD ha trascendido su dimensión institucional para convertirse en un símbolo de resistencia y reflexión sobre el estado de la salud pública en Colombia. La abundancia y diversidad de registros visuales evidencia la necesidad de analizar cómo estas representaciones contribuyen tanto a la construcción de memoria colectiva como a la comprensión de fenómenos sociales complejos desde la perspectiva del pensamiento visual y la creación interactiva.

La implementación de la Ley 100 tuvo como una de sus principales consecuencias la agudización de la crisis en los hospitales públicos, al permitir una retirada progresiva del Estado frente a sus responsabilidades con la salud de los colombianos y dejar a la deriva las entidades de carácter oficial \parencite{Castiblanco2017}. Diversos estudios han documentado cómo, tras la publicación de esta ley en 1993, el hospital entró en una rápida fase de decadencia.

Esta situación provocó la pérdida de hogares y bienes materiales de numerosos empleados y sus familias, lo que derivó en la "toma" del Hospital. Paralelamente, este hecho marcó el inicio de una extensa batalla jurídica que, incluso en 2015, más de catorce años después, continuaba vigente \parencite{Orlando2015}. La lucha por los derechos laborales negados durante la precipitada e insatisfactoria liquidación de la Fundación San Juan de Dios se ha extendido ya por dos décadas.

Es importante señalar que, según la documentación institucional, el hospital nunca ha dejado de existir formalmente. De hecho, al momento de redactar este informe, se encuentra en desarrollo un proyecto de «intervención integral de 7 de los 17 edificios de mayor valor patrimonial del Complejo Hospitalario, así como de los espacios emblemáticos del costado nororiental, con el fin de consolidar la primera etapa de la reactivación funcional de este hospital».