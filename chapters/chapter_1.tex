\section*{Introducción al fenómeno}

La crisis del HSJD emerge como un laboratorio social donde el diseño y la creación interactiva pueden explorar las dinámicas de transformación institucional, memoria colectiva y resistencia ciudadana. Más allá de una indagación entre registros documentales o artísticos, este estudio pretende señalar cómo diversas acciones simbólicas, argumentales, documentales y estéticas evidencias de una intensa atracción visual en torno a la crisis del HSJD, los registros de estas acciones nos brindan un corpus de imágenes con potencial para interpretar los síntomas visuales de la crisis sistémica, así como los deterioros arquitectónicos y patrimonials en juego con el drama humano y social.

El caso del Hospital San Juan es emblemático dentro del contexto de las crisis sociales e institucionales generadas por los cambios masivos en los sistemas de salud orientados al servicio público. ¿Cómo comprender, explicar e interpretar los registros visuales y las obras que serán tratadas como \guillemotleft evidencias\guillemotright\ que articulan el drama humano y la defensa de un bien público?. 

A estas evidencias las denominamos imagen-síntoma, una presencia disruptiva en la imagen que interrumpe el curso normal de la representación: supervivencias, latencias y reapariciones que habitan las imágenes. La metodología aplicada a la colección visual de memoria social permite abordar el fenómeno a través de la imagen.

El concepto de imagen-síntoma\footnote{\begin{quote}La idea de la imagen como síntoma es lo que quiere tomar Didi-Huberman de Freud y aplicarlo al campo de la historia del arte. Por supuesto, no sobra decir que se trata de dos disciplinas muy distintas, que la experiencia de las imágenes oníricas no es igual a la de las imágenes artísticas. Aun así, el uso del carácter sintomático de la imagen en la historia del arte no es tan diferente al del análisis de los sueños. Sólo que aquí, en lugar de evocar lo onírico, Didi-Huberman lo utilizará para nombrar esa perturbación que lo visual causa dentro de lo visible. \parencite[p. 37]{VegaArevalo2017}\end{quote}} resulta fundamental para comprender cómo las representaciones visuales del HSJD trascienden la mera documentación para convertirse en manifestaciones de tensiones sociales más profundas. Según Didi-Huberman, estas imágenes actúan como síntomas que revelan conflictos latentes en el tejido social.

\begin{quote}
Si la imagen es un síntoma -en el sentido crítico y no clínico del término-, si la imagen es un malestar en la representación, es porque indica un futuro de la representación, un futuro que no sabemos aún leer, ni, incluso, describir. \parencite[p. 177]{DidiHuberman2011}
\end{quote}

Este estudio indaga en el papel fundamental de la imagen para la contemplación profunda de los fenómenos sociales, ofreciendo nuevas perspectivas para comprender su complejidad y contribuyendo al debate sobre el papel del arte y el diseño en los procesos de cambio social y la construcción de memoria colectiva.

El diseño, entendido como práctica crítica y performativa, permite desvelar las capas de significado ocultas en los registros visuales. No se trata solo de documentar, sino de construir narrativas que revelen las tensiones sociales subyacentes. Las imágenes del HSJD se convierten así en interfaces de memoria, donde cada fragmento, cada ruina, cada registro artístico funciona como un nodo de información compleja.

La creación interactiva se presenta como sugerencia para trascender la observación pasiva, invitando a una participación activa en la construcción de sentido. En este contexto, las obras artísticas y los registros visuales no son meros documentos, sino dispositivos de activación memorial que permiten:

\begin{itemize}
    \item Desarticular narrativas oficiales sobre el abandono institucional
    \item Visibilizar las experiencias de los trabajadores y comunidades afectadas
    \item Generar nuevas formas de comprensión y elaboración del conflicto social
\end{itemize}

\section*{Contextualización histórica}

El Hospital San Juan de Dios de Bogotá fue fundado en 1723 como un centro de atención médica y asistencia social para los más necesitados. A lo largo de los siglos, el hospital se convirtió en un referente de la salud pública en Colombia, atendiendo a miles de pacientes y formando a generaciones de profesionales de la salud. Sin embargo, a partir de la década de 1990, el HSJD comenzó a experimentar una serie de crisis institucionales y financieras que llevaron a su cierre.

En el año 2001, coincidiendo con la salida del último paciente, “3.640 personas quedaron desempleadas, y aproximadamente 1.500 trabajadores no recibieron la liquidación correspondiente por sus años de servicio” \parencite{Castiblanco2017}. No obstante, este hecho fue solo uno de los múltiples detonantes de las acciones de resistencia que se han desarrollado a lo largo de los años en torno a este caso. Los afectados directos emprendieron diversas formas de lucha y resistencia frente a la pérdida de sus empleos y de un espacio vital para su realización personal y profesional. Estas, sin embargo, no fueron las únicas motivaciones ni las únicas comunidades que centraron su atención en el complejo entramado sistémico de problemáticas sociales asociado al caso.

Ese hospital ya estaba en estado de malestar mucho antes de la salida del último paciente. De acuerdo con el profesor Mario Hernández investigador en historia de la medicina, en las décadas de los 70 y 80 inició la crisis de los Estados de Bienestar, que derivaron en acciones institucionales del pensamiento llamado neoliberal. Así, la dinámica económica global de tratar de buscar que los servicios de interés público como la salud fueran parte de las dinámicas de mercado condujeron al San Juan hacia un proceso de transformación. El Hospital que nació como beneficencia trató fallidamente de adaptarse a las nuevas demandas neoliberales, en consecuencia dejó de disponer tiempo o dinero para cuidar el patrimonio arquitectónico. El hospital se estaba descascarando mucho antes del embate de la Ley 100 y aún así se mantendría vivo y funcional hasta su último aliendo, cumpliendo su propósito de atención y cuidado de la salud.

Paralelamente a los esfuerzos institucionales para reabrir el San Juan, surgieron experiencias de lucha social, señalamientos estéticos y acciones memoriales que abordaron la complejidad del fenómeno. Esta investigación explora cómo se construye sentido a través del registro de obras e imágenes artísticas, estéticas, poéticas y no funcionales.

Fortalecer la conexión entre la contextualización histórica y el enfoque en las prácticas estéticas permite comprender cómo las imágenes del HSJD revelan tensiones sociales subyacentes y generan nuevas formas de comprensión y elaboración del conflicto social.

\subsubsection*{Laboratorio social}

\subsubsection*{Exhibición y creación}

\subsubsection*{Control institucional}


\section*{Planteamiento del problema }
¿Ante el abandono qué hacemos? ¿nos quedamos, nos vamos, luchamos, por qué, …hasta cuando? En el año 2001 salió el último paciente del HSJD, un hito, sin embargo, ese día no se inhabilitó el San Juan. El abandono ya había comenzado años antes. Las gestiones institucionales para su liquidación y reapertura total han durado décadas, y a la fecha no es posible afirmar que el San Juan haya cerrado o abierto definitivamente.

Se presenta una contextualización espacial e histórica para darle sentido a la revisión de los signos de las resistencias, registros gráficos y audiovisuales de los procesos de creación artística, instalaciones in-situ, performativas y de dramaturgia, que, como veremos en el desarrollo de este informe, contienen signos icónicos e indicios de la imaginación social. En el año 2007 aparece una de las primeras manifestaciones artísticas que señalan explícitamente la problemática del HSJD, de allí en adelante y hasta el año 2016 serán producidas y exhibidas una docena de obras artísticas hechas en torno a esta situación.

Las imágenes analizadas a la luz del concepto imagen-síntoma tienen una carga de resistencia al olvido, no en si mismas, sino en aquello que cargan o permiten, son imágenes con las cuales es posible vislumbrar ruinas que son síntomas del drama humano y social, en el corpus de imágenes analizadas se evidencian recurrencias en el discurso desde distintos ámbitos de participación en el señalamiento a la crisis del San Juan. Las imágenes de registro, y respuestas críticas frente a una situación o evento de crisis nos exigen más de lo que explican, en ellas devienen explicaciones en aparente discordia.


\section*{Justificación y relevancia}
La crisis del Hospital San Juan de Dios HSJD representa una problemática sistémica donde convergen múltiples dimensiones interconectadas e interdependientes: económicas, administrativas, sociales, políticas y culturales. Desde la perspectiva del diseño y la creación interactiva, resulta pertinente abordar estas problemáticas sociales complejas con enfoques y metodologìas innovadoras, especialmente en el actual contexto colombiano de construcción de paz, donde la transformación social requiere trascender la resolución del conflicto armado para atender otros ámbitos del bienestar social \parencite[p. 313]{Capra1998}\footnote{Señala Capra que \textit{"El principio de flexibilidad sugiere también una correspondiente estrategia de resolución de conflictos. En toda comunidad aparecen inevitablemente discrepancias y conflictos que no pueden ser resueltos en favor de una u otra parte."}}.

El caso del HSJD emerge como un fenómeno paradigmático donde la movilización ciudadana ha mantenido vigente el debate público sobre la salud como derecho fundamental. Esta crisis institucional se ha convertido en un catalizador para la reflexión crítica y la resistencia social, manifestada a través de diversas prácticas disciplinares y expresiones simbólicas. El deterioro físico y social del hospital ha generado un particular interés visual, que la curadora Ana María Lozano describe como un estado de "letargia intermedia entre la muerte total y una suspensión cada vez más declinante".

La investigación se fundamenta en un corpus documental de más de 1,000 registros visuales, seleccionados de producciones académicas y artísticas que abordan directamente la crisis y cierre del HSJD. Es significativo señalar la correlación temporal entre los momentos más álgidos de la crisis hospitalaria y el incremento en la producción artística, coincidiendo además con una intensificación en la cobertura mediática del caso en los principales medios de comunicación bogotanos.

Este fenómeno visual-documental evidencia cómo la crisis del HSJD ha trascendido su dimensión institucional para convertirse en un símbolo de resistencia y reflexión sobre el estado de la salud pública en Colombia. La abundancia y diversidad de registros visuales sugiere la necesidad de analizar cómo estas representaciones contribuyen a la construcción de memoria colectiva y a la comprensión de fenómenos sociales complejos desde la perspectiva del pensamiento visual y la creación interactiva.

Entre las múltiples consecuencias de la implementación de la Ley 100, encontramos el aumento de la crisis de los hospitales públicos, pues permitió una retirada progresiva del Estado frente a sus responsabilidades con la salud de los colombianos y dejó a la deriva las entidades de carácter oficial \parencite{Castiblanco2017}.

Sin lugar a duda y como lo describen varios estudios que se han hecho al respecto, algunos de ellos citados a lo largo de este informe, después de publicación de la Ley 100 de 1993 el hospital entro en rápida decadencia.

Esto tuvo como consecuencia la pérdida de hogares y bienes materiales de varios empleados y sus familias, lo cual finalizó en la “toma” del Hospital, pero, por otro lado, fue el principio de una larga lucha jurisprudencial que aún, en el 2015, más de catorce años después, sigue en pie \parencite{Orlando2015}.

Ya van dos décadas de lucha por esos derechos laborales que fueron negados durante una atropellada e insatisfactoria liquidación de la Fundación San Juan de Dios. 

En la historia documental institucional se verá que el hospital en ningún momento desaparece, de hecho, a la fecha de escritura de este informe existe un proyecto de «intervención integral de 7 de los 17 edificios de mayor valor patrimonial del Complejo Hospitalario, así como de los espacios emblemáticos del costado nororiental, con el fin consolidar la primera etapa de la reactivación funcional de este hospital» .

\section*{Preguntas de investigación}
En paralelo a estos grandes esfuerzos institucionales, en el HSJD continúan las experiencias de lucha social, investigación y señalamiento artístico sobre la complejidad del fenómeno, a la par de la desmaterialización funcional del hospital se han realizado acciones de construcción de sentido en sus ámbitos patrimoniales, estéticos, poéticos y no funcionales. ¿En este contexto que se entiende por imagen artística?, ¿cómo las imágenes expresan la situación de crisis y resistencia en el HSJD de Bogotá?