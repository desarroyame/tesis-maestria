\vspace{4cm}
        
Espacio para las escenas 



\clearpage
\section{Conclusiones}

El montaje permite escenificar la emergencia de temporalidades, enunciaciones, anacronismos y otros síntomas en el conjunto de registros artísticos, imágenes documentales y textos visuales. A través de este proceso, se revela el entramado de atemporalidades y se genera una experiencia de sentido mediante los pasajes simbólicos del caso de estudio. El montaje constituye, entonces, un análisis interpretativo sobre la selección de imágenes que expresa espacial y temporalmente en <<cuadros>> el entramado visual y social, creando un ambiente de interpretación exploratoria que deriva en experiencias de construcción de sentido.

Denominamos escenificación tanto a la experiencia como al resultado del montaje utilizando las imágenes seleccionadas. El ordenamiento de las escenas representa una forma espacial de montaje que permite unir lo comprensible con lo imaginario, generando experiencias de sentido que confrontan nuestra mirada con el entramado sistémico de esta compleja situación social.

Las escenas materializan la intersección conceptual entre la antropología de la imagen y el método socio semiótico. La experiencia de inmersión en el recorrido visual sobre el San Juan de Dios ofrece posibilidades interpretativas que impulsan el deseo de comprensión, guiando al observador hacia la revelación de los entramados emergentes en este discurso no textual. Se omite deliberadamente el enfoque historicista, pues el objetivo es resaltar la existencia del anacronismo sin necesidad de una enunciación secuencial de acontecimientos.

El fenómeno sociocomunicativo del caso de estudio se manifiesta a través de elementos anacrónicos que atraviesan la imagen no-domesticada - aquella que se desprende de propósitos publicitarios o periodísticos - y se integra en redes dialógicas no-textuales mediante el montaje con otros discursos visuales. Esta interacción facilita la emergencia de la imagen-síntoma y los indicios de anacronismos.

La metodología propuesta permitió materializar objetos mediales para la conceptualización. El ``juego'' con las imágenes funcionó como estrategia de trabajo para abordar ideas abstractas, asociarlas y organizarlas espacialmente.

El montaje intencionado para la construcción de sentido enfatiza los síntomas visuales del discurso, generando conjuntos y redes significantes e imaginarias, interpretables desde los fundamentos conceptuales de la antropología de la imagen y la socio semiótica. Estas redes significantes o "fascinaciones" de la mirada subrayan la atracción irresistible del observador, quien en su gesto de registro o creación de imagen materializa enunciaciones emocionales y de sentido.

Un aspecto recurrente es el encuadre de lugares y objetos deteriorados por el tiempo: las imágenes evidencian el triunfo del agua, el musgo y la hierba sobre las superficies. Esta característica visible contrasta con el impulso de quienes han resistido y cuidado el San Juan, personas que han luchado a contracorriente mediante actos de expresión visualmente documentados, generando impulsos de resistencia ante la hostilidad del contexto y las circunstancias.
