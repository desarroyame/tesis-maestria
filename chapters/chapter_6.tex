\section{Escenificación: montaje de imágenes}

Mediante el montaje se escenifica la emergencia de temporalidades, enunciaciones, anacronismos y otros síntomas en el conjunto de registros de obra de arte, imágenes documentales y demás textos visuales, además, revelar el entramado de atemporalidades y encaminarnos hacia una experiencia de sentido a través de los pasajes simbólicos en el caso de estudio. Montaje es entonces el análisis interpretativo sobre la selección de imágenes, que expresa espacial y temporalmente (en escenas) el entramado visual y social. Se escenifica un ambiente de interpretación exploratoria de las imágenes que derive en experiencias de construcción de sentido.

A la experiencia y resultado del montaje, usando las imágenes seleccionadas, le llamaremos escenificación. Ordenar las escenas es una forma espacial de montaje, para juntar lo comprensible con lo imaginario, y crear así experiencias de sentido que confronten nuestra mirada con el entramado sistémico de esta situación social compleja.

Las escenas materializan la intersección conceptual de dos marcos metodológicos: la antropología de la imagen y el método socio semiótico. Tener la experiencia de inmersión en el camino de imágenes sobre el San Juan de Dios brinda posibilidades interpretativas, para impulsar el deseo de comprensión encaminando al observador hacia la revelación de los entramados emergentes en este discurso no textual, omitiendo el enfoque historicista dado que se quiere resaltar la existencia de anacronismo no es necesario hacer enunciación secuencial de acontecimientos.

Sintetizando, el fenómeno socio comunicativo en torno al estudio de caso se manifiesta a través de los elementos anacrónicos que atraviesan la imagen no-domesticada, es decir la imagen que que se aísla de sus propósitos publicitarios o de reportería, y se pone en redes dialógicas no-textuales mediante el montaje con otros discursos visuales, esta interacción propicia la emergencia de la imagen-síntoma y los indicios de anacronismos.

El método propuesto permitió la materialización de objetos mediales para la conceptualización, “jugar” con las imágenes fue una estrategia de trabajo que permitió abordar ideas abstractas para asociarlas y organizarlas espacialmente (Abril, 2007, p. 79).

El montaje intencionado para la construcción de sentido destaca los síntomas visuales del discurso, produce conjuntos y redes significantes e imaginarias, interpretables en base a los sustratos conceptuales de la antropología de la imagen y la socio semiótica. Estas redes significantes o “fascinaciones” de la mirada, enfatizan la irresistible atracción de quién observa y que en su gesto de registro o creación de imagen materializa enunciaciones emocionales y de sentido.

Un aspecto recurrente es el encuadre en los lugares y objetos carcomidos por el paso del tiempo, en las imágenes se hace visible el triunfo del agua, el mugo y la hierba sobre sus superficies, esta característica o atributo visible en la imagen contrasta con el impulso de las personas que han hecho resistencia y han cuidado San Juan, estas personas han luchado a contracorriente creando actos de expresión que se evidencian visualmente, creando impulsos de resistir ante la hostilidad del contexto y las circunstancias.

\subsection{Montajes}