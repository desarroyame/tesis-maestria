La perspectiva desarrollada por Didi-Huberman, en diálogo con los planteamientos de Walter Benjamin y Aby Warburg, establece que la observación de una imagen genera significado sobre lo observado. Para los propósitos de esta investigación, es necesario definir los elementos teóricos e instrumentales que se emplearán en la "lectura" del conjunto de imágenes seleccionado. El campo de los estudios visuales proporciona un marco argumentativo para trabajar con imágenes, lo que esencialmente implica pensar a través de ellas.


Como señalan \parencite{Perez2010}:
\begin{quote}
Trabajar con imágenes no sólo es muy entretenido, sino que el proceso de encontrarlas y superponerlas es también muy esclarecedor intelectualmente. Muchas veces primero encuentro la imagen y luego escribo el texto que la acompaña. (p. 45)
\end{quote}

El proceso intelectual de esta investigación comienza con la búsqueda y selección de imágenes. Considerando el extenso impacto mediático del caso de crisis y cierre del Hospital San Juan de Dios (HSJD), se ha realizado una cuidadosa selección que contribuye a la definición de categorías de análisis y codificación.

Esta investigación busca identificar, mediante recursos metodológicos y conceptuales propios del diseño y el arte, una aproximación rigurosa al caso de crisis y cierre del HSJD. El caso del San Juan resulta emblemático en el contexto de las crisis sociales e institucionales derivadas de las transformaciones en los sistemas de salud pública, destacándose por su trayectoria histórica como institución hospitalaria fundamental para Bogotá y referente en la medicina suramericana.

La abundante producción investigativa y comunicativa evidencia su relevancia histórica, legando un importante acervo de información sobre aspectos históricos, sociales, políticos, laborales, médicos y pedagógicos. Sin embargo, esta investigación se centra específicamente en las expresiones artísticas como constructoras de sentido. Los registros de obra y enunciados visuales se analizan como evidencias que, según \parencite{Montejo2003}, "articulan la vida personal y cotidiana con la defensa de un bien público como es el Hospital San Juan de Dios" (p. 78).

Para delimitar el alcance del estudio, es importante precisar que no nos limitaremos a un único tipo de imagen. Como señala el iconólogo \parencite{Mitchell2005}, el término "imagen" denota tanto el componente físico u objetual como entidades mentales, memoriales y perceptuales. Aunque las imágenes mentales carecen de materialidad física, tienen existencia en nuestro cuerpo y mente, manifestándose a través del lenguaje en el colectivo social.

Las categorías de análisis para las imágenes artísticas relacionadas con la crisis del HSJD incluyen conceptos como: imagen-síntoma, imagen-malicia, imagen-combate, imagen-aura, imagen-fantasma \parencite{DidiHuberman2011}; imagen-virtual, imagen-digital \parencite{Manovich2005}; imagen-dialéctica \parencite{Benjamin2004}; e imagen-tiempo, imagen-movimiento, imagen-recuerdo, imagen-sueño \parencite{Deleuze1985}.

La metodología empleada trasciende la mera descripción iconográfica, incorporando el análisis iconológico para descifrar mensajes subyacentes. Para profundizar en las relaciones de sentido entre la realidad social y las imágenes artísticas del HSJD, se utiliza el concepto de \textit{montaje} desarrollado por \parencite{Benjamin2004} y la noción de \textit{supervivencia} de \parencite{Warburg2010}, revelando latencias y síntomas de la memoria social a través de la experiencia visual del observador.

Esta investigación refiere también a la metodología de Investigación Basada en Artes, mediada por la antropología de la imagen. La postura investigativa adopta el pensamiento a través de imágenes y sus vehículos, incluyendo la imaginación, los objetos materiales y los medios virtuales \parencite{Leavy2018}.

\section{Imagen-síntoma}

Delimitemos un marco de referencia para examinar las imágenes artísticas relacionadas con el HSJD. El objetivo es construir un escenario que invite al observador a ordenar visualmente los atributos de las imágenes, permitiendo que la mirada incorpore valores de sentido no-textuales sobre este fenómeno sociocultural.

Al examinar el conjunto de manifestaciones artísticas y visuales relacionadas con el HSJD, se evidencian emergencias visuales que funcionan como metáforas de patologías, materializando los síntomas de un momento de crisis. Para analizar las relaciones y el valor de sentido en la imagen sobre este fenómeno sociocultural, es necesario establecer las conexiones entre la construcción de sentido y la imagen, lo que denominamos imagen-síntoma.

La imagen-síntoma se caracteriza por ser una conjunción singular entre la diferencia y la repetición, manifestándose como una irrupción en el curso normal de la representación. Esta comprende las supervivencias, latencias y reapariciones que habitan las imágenes. Como señala \parencite[p.~304]{DidiHuberman2011}:

\begin{quote}
¿Qué es, en efecto, un síntoma si no el signo inadvertido, no familiar, a menudo intenso y siempre disruptivo, que anuncia visualmente algo que no es todavía visible, algo que todavía no conocemos? Si la imagen es un síntoma -en el sentido crítico y no clínico del término-, si la imagen es un malestar en la representación, es porque indica un futuro de la representación, un futuro que no sabemos aún leer, ni, incluso, describir.
\end{quote}

En su propuesta teórica, Didi-Huberman desarrolla otras caracterizaciones para estas imágenes donde perviven y se proyectan "síntomas". Un ejemplo es la imagen-fantasma, configurada a partir del concepto warburguiano de \textit{Nachlebem} o supervivencia. Como señala \parencite{Dieguez2013}, estas imágenes se manifiestan como síntomas para reaparecer en otros tiempos, adquiriendo un carácter profético.

\subsection*{Sentido crítico y no clínico}

\section{Anacronismo}

El anacronismo, entendido como la intrusión de una época en otra, constituye un concepto fundamental en la antropología de las imágenes desarrollada por \parencite{DidiHuberman2011}. Este enfoque nos permite abordar la compleja temporalidad inherente a las imágenes, exponiéndolas a interpretaciones insospechadas y lógicas no convencionales. La identificación y análisis de estos anacronismos se convierte en una herramienta metodológica esencial para develar el sentido a través de las huellas de la memoria social manifiestas en la imagen artística.

Como señala Didi-Huberman: "el anacronismo parece surgir en el pliegue exacto de la relación entre imagen e historia; las imágenes, desde luego, tienen una historia; pero lo que ellas son, su movimiento propio, su poder específico, no aparece en la historia más que como un síntoma -un malestar, una desmentida más o menos violenta, una suspensión" \parencite[p. 48]{DidiHuberman2011}.

Al situarnos ante una imagen, nos encontramos simultáneamente ante un tiempo no cronológico. La temporalidad en la imagen reside en nuestra imaginación y trasciende la secuencialidad convencional. La dimensión memorativa de la imagen se proyecta en el inconsciente y se manifiesta a través de superposiciones temporales, revelando así atributos fundamentales de la memoria social.

\parencite{DidiHuberman2011} sostiene que el choque de temporalidades en la imagen libera todas las modalidades del tiempo mismo, elaborando una paradoja: mientras la imagen en la historia se dispersa, también se cristaliza en obras concretas. Las imágenes contienen frágiles supervivencias que nos conmueven y permiten una comprensión no verbal de los fenómenos. Por lo tanto, al desmontar el registro de obra plástica, documental, fotográfica, dramatúrgica o performativa de su función y contexto de producción original, podemos revelar aspectos del fenómeno que trascienden los motivos iconográficos evidentes.

