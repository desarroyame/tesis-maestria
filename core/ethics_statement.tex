\section*{Declaración de ética}
\setlength{\parskip}{1em}

\textcolor{edit30sept}{Yo, Juan Carlos Arroyo Sosa, quien suscribe este texto,} he sido durante años un silencioso observador \textcolor{edit30sept}{de la crísis, cierre y transformación del Hospital San Juan de Dios}, y como afectado contiguo al drama social derivado de la crisis y fin de los servicios hospitalarios, he visto los síntomas de decadencia funcional, deterioro de las edificaciones y el drama humano vivido por las personas alrededor de la persistencia en las distintas formas de resistir al fin del HJSD. Una de esas trabajadoras a las que aún se les adeuda su liquidación, es mi madre, la enfermera Berenice Sosa o como aún después de tantos años de no ejercer su profesión, sus compañeras del San Juan le siguen diciendo “jefe Berenice”.

Esta investigación titulada \textit{Significación social y supervivencia simbólica en la producción de imagen artística sobre la resistencia y crisis del Hospital San Juan de Dios, Bogotá, 2007-2017} se ha realizado respetando los principios éticos fundamentales de integridad, transparencia y respeto hacia las personas y los materiales involucrados.  

La recolección de datos incluyó el uso de material de registro de obra artística, consulta bibliográfica, así como registro fotográfico personal y cedido temporalmente para consulta por ex trabajadores del hospital, quienes colaboraron de manera voluntaria y con su consentimiento informado. Se garantizó la confidencialidad de los participantes y el uso adecuado de los recursos proporcionados, asegurando que sus contribuciones fueran tratadas con el mayor respeto y se emplearan exclusivamente con fines académicos y de investigación.  

El estudio busca promover la comprensión de las dimensiones simbólicas y sociales de la resistencia y crisis del Hospital San Juan de Dios, contribuyendo a preservar su memoria histórica. Se evitó cualquier manipulación de la información recopilada y se actuó con responsabilidad frente a los derechos de los artistas, autores y participantes involucrados en este proceso.  

\pagebreak