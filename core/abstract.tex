\section*{Resúmen}
\setlength{\parskip}{1em}

El propósito de esta investigación es la construcción de sentido a través de la imagen sobre el caso de crisis y resistencias del Hospital San Juan de Dios, Bogotá. Se propone apelar a la imagen como vestigio y sustrato dialógico del imaginario social sobre las profundas transformaciones en el sistema público de salud colombiano que ocasionaron el entramado sistémico de problemáticas sociales complejas. Es una crisis en la cual al menos 3.640 personas quedaron desempleadas, y aproximadamente a 1.500 trabajadores no se les pagó la liquidación por sus años de trabajo, con severas consecuencias en el bienestar de estas personas, desencadenando además un drama y abandono social y estructural que no fue ajeno a la mirada artística, cultural y patrimonial.

Esta investigación se sustenta en los campos antropología de la imagen, estudios visuales y socio semióticos, para crear un fundamento argumentativo que permita «trabajar» con imágenes, que es en esencia, pensar a través de las imágenes.

El alcance es apropiar prácticas metodológicas que permitan el análisis de aspectos del fenómeno de crisis, a través de la interpretación de imagen. Así, se hace una apropiación de la idea de montaje complementado con algunos pasos de enunciación de imaginarios sociales y estructuras semióticas llamadas imágenes-síntoma y anacronismos, para luego recurrir a la capacidad del observador como mediador entre la imagen y el discurso.

\vspace{1cm}
\textbf{Palabras clave:} imagen-síntoma, resistencia social, memoria colectiva, crisis hospitalaria, semiótica visual, antropología de la imagen.
\pagebreak

