\section*{Resumen}
\setlength{\parskip}{1em}

Esta investigación aborda los desafíos de las trayectorias convertidas en imagen relacionadas con los derechos laborales del arquetipo de enfermera del Hospital San Juan de Dios (HSJD), enfocándose en la experiencia personal de mi madre durante la crisis del hospital como \textit{wicked problem}. La pregunta central que guía este estudio es: ¿Cómo la imagen artística sobre la crisis y resistencia del HSJD entre 2000 y 2015 contribuye a la significación y supervivencia simbólica de este evento en la memoria colectiva?.

De acuerdo con las referencias bibliográficas y audiovisuales consultadas, la belleza estética del hospital, el atractivo de la ruina y la complejidad histórica han generado múltiples perspectivas de creadores visuales, artistas, periodistas e investigadoras académicas que han observado y documentado el HSJD. \textcolor{edit30sept}{Sin embargo, esta prolífica producción no ha logrado generar para las enfermeras satisfacción en términos de justicia laboral y simbólica. La situación se presenta como un entramado complejo, lleno de expectativas fallidas, en el que al menos 3.640 personas quedaron desempleadas y aproximadamente 1.500 trabajadores no recibieron su liquidación.}

La investigación se desarrolla como una meta-composición estética y documental que trasciende la mera documentación o el análisis sociopolítico. Se sustenta en los campos de la antropología de la imagen, estudios visuales y sociosemióticos, para crear un fundamento argumentativo que permita «trabajar» con imágenes, que es, en esencia, pensar a través de las imágenes.

Metodológicamente, se operacionaliza el análisis mediante la apropiación de la idea de montaje, complementada con la identificación de imaginarios sociales y estructuras semióticas denominadas imágenes-síntoma y anacronismos. Esta aproximación permite interpretar cómo diversas acciones simbólicas, argumentales y estéticas evidencian una intensa producción visual en torno a la crisis del HSJD.

Los resultados revelan que los registros de estas acciones conforman un \textcolor{edit30sept}{ \emph{corpus}} de imágenes que permite interpretar tanto los síntomas visuales de la crisis sistémica como el deterioro arquitectónico y patrimonial en su interacción con el drama humano y social. La investigación demuestra cómo las representaciones visuales del HSJD trascienden la mera documentación para convertirse en evidencias del drama humano y la defensa de un bien público. \textcolor{edit30sept}{La tesis expone, mediante la metodología del montaje y la meta-composición estética y documental, la disrupción del fenómeno del San Juan, mostrando cómo estas operaciones visuales ofrecen oportunidades de reconfiguración de sentido en la memoria colectiva.}

Este estudio contribuye al campo del diseño y creación interactiva al proponer una metodología específica para el análisis de memoria social a través de la imagen, donde el montaje artístico se integra como medio de exploración y validación de hipótesis sobre la supervivencia simbólica de eventos críticos en el imaginario colectivo.

\vspace{1cm}
\textbf{Palabras clave:} \textcolor{edit30sept}{Meta-composición, montaje, imagen-síntoma, semíotica visual, antropología de la imagen, crisis hospitalaria}   
\pagebreak
